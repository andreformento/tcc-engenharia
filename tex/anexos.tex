\begin{anexosenv}

    % Imprime uma página indicando o início dos anexos
    \partanexos

    \chapter{Especificação de hardware do servidor}\label{anexo-especificacao-de-hardware-do-servidor}

        \index{alíneas}\index{subalíneas}\index{incisos}As especificações de Hardware
        do computador que foram feitos os testes estão descritas nos próximos tópicos.
        Os testes foram realizados em um notebook de uso pessoal.

        \begin{alineas}

        \item CPU

        \begin{alineas}

            \item Arquitetura: x86\_64

            \item Empresa: Intel Corp.

            \item Produto: Intel(R) Core(TM) i7-4.800MQ CPU @ 2,70GHz

            \item Quantidade de núcleos: 8

            \item Frequência padrão: 2.848MHz

            \item Frequência mínima:  800MHz

            \item Frequência máxima: 3.700MHz

            \item Virtualização: VT-x

            \item L1d cache: 32K

            \item L1i cache: 32K

            \item L2 cache: 256K

            \item L3 cache: 6.144K

        \end{alineas}

        \item Memória

        \begin{alineas}

            \item Capacidade por unidade: 8.192 MB

            \item Quantidade: 2

            \item Capacidade total: 16.384 MB

            \item Modelo: SODIMM

            \item Set: None

            \item Tipo: DDR3

            \item Velocidade: 1.600 MT/s

        \end{alineas}

        \item Armazenamento

        \begin{alineas}

            \item Modelo: Samsung SSD 840

            \item Empresa: Samsung

            \item Device: SSD 840

            \item Detalhe: Samsung SSD 840 PRO Series (DXM05B0Q)

            \item Capacidade: 256 GB

        \end{alineas}

        \item Sistema operacional

        \begin{alineas}

            \item Sistema operacional: Linux

            \item Distro: Fedora

            \item Release: 26

            \item Versão do Kernel: 4.13.12-200.fc26.x86\_64

        \end{alineas}

        \end{alineas}

    %\chapter{Outro anexo}

    %escreve outro anexo

\end{anexosenv}
