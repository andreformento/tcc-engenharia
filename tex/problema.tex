\chapter{Problema}

Atualmente a venda de ingressos para eventos ocorre em websites.
O comum é que haja muita divulgação na mídia a respeito do evento.
A venda de ingressos costuma ser muito disputada devido a limitação
física que o local onde ocorrerá permite, assim como, a oferta de
lugares privilegiados, ingressos com descontos (estudante, idoso), etc.
Desta forma, muitas pessoas buscam os ingressos logo após o início das
vendas, gerando uma demanda muito alta nas primeiras horas de venda.

Conforme \cite{10-motivos-para-vender-online}, os compradores consideram
vários motivos para fazer sua compra online, como comodidade por não
precisar ir até um ponto de venda; não enfretam fila; segurança; etc.

\index{alíneas}\index{subalíneas}\index{incisos}A venda de ingressos online
tem diversas temas a serem tratados. Porém, este documento abordará a parte
técnica do sistema focando nos seguintes itens que atenderá alta demanda:

\begin{alineas}

  \item Performance

  \begin{alineas}
     \item sites que não estão preparados para ter grande variação de tráfego
           tendem a ter problemas de performance ou lentidão
  \end{alineas}

  \item Escalabilidade

  \begin{alineas}
     \item a
  \end{alineas}

  \item Disponibilidade

  \begin{alineas}
     \item quando não planejada, a alta demanda inesperada pode causar até mesmo
           indisponibilidade do sistema, fazendo com que fique temporariamente
           fora do ar
     \item indisponibilidade causa perda de vendas. Segundo
           \cite{disponibilidade-downtime-perdas}, uma queda no serviço S3 da
           Amazon gerou um prejuízo estimado de \$160.000,00 somente nas empresas
           da S\&P 500.
  \end{alineas}

\end{alineas}
