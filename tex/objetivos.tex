\chapter{Objetivo}

O objetivo principal é tornar viável a arquitetura de um sistema web para atender
alta demanda de requisições de um site houver aumento de demanda.
Por exemplo, no momento da abertura das vendas de ingressos para um determinado evento.
Também é necessário que haja otimização do sistema quando estiver ocioso.
Ou seja, reduzir a quantidade de recursos utilizados quando não houver
grande demanda - reduzindo o custo.

\section{Escalabilidade e microserviços}
A escalabilidade é a parte mais relevante desse projeto. Tornar um sistema escalável
é um grande desafio. Para ser possível, vários pontos da arquitetura devem ter a
abordagem adequada. Por exemplo, a forma como o sistema é divido implica em vários
fatores. A abordagem de microserviços evita que haja um ponto único de falha. A
utilização do conceito de microserviços possibilitará a escalabilidade de maneira
simplificada, rápida e de acordo com o que é visto atualmente no mercado.

Um sistema de vendas de ingressos online é abrangente e envolve divulgação,
exibição de eventos, lugares e valores disponíveis, reserva do ingresso -
que é limitada, assim como a venda e suporte pós-venda. Um sistema completo possui essas
e outras características. Do ponto de visto técnico, várias das abordagens discutidas
podem ser utilizadas em muitos desses pontos. Porém, a parte prática será desenvolvida
apenas em relação ao serviço de backend que atentererá requisições do contexto de reserva
de ingressos.

\section{Implementação da reserva de ingressos}

A parte prática o trabalho irá tratar especificamente o contexto de reserva de ingressos.
Este serviço de reserva de ingressos será uma forma prática para aplicação
de serviço de alta disponibilidade. Serão abordadas técnicas de microserviços.
Com isso, vários dos itens abordados poderão ser aplicados em outras partes de um sistema
de venda de ingressos em formato de serviços que terão os mais diversos contextos.

\section{Infraestrutura para rodar o serviço}

O serviço será uma aplicação que rodará em um ambiente Linux através
de uma imagem Docker e poderá ser replicada com Kubernetes. Todas as essas ferramentas
serão abordadas, assim como, técnicas de programação voltadas a seriços de alto
desempenho. A aplicação terá uma API REST que responderá requisições HTTP.

A sugestão é que essa infraestrutura inteira rode em alguma plataforma em nuvem para
que se obtenham os recursos necessários para que seja possível obter alta escalabilidade,
performance e disponibilidade necessária no momento da alta demanda da venda de ingressos.

\subsection{Ferramentas de infraestrutura}

\index{alíneas}\index{subalíneas}\index{incisos}Ter vários microserviços implica em gerenciar
vários sistemas. E isso significa que automação é fundamental para que isso funcione
\cite{martin-fowler-microservices}.
Para simplificar o gerenciamento dos serviços serão utilizadas algumas ferramentas que
irão facilitar a manutenção, escalabilidade e monitoramento:

\begin{alineas}

  \item Docker

  \begin{alineas}
     \item Para criação, teste e implantação de aplicações rapidamente \cite{aws-o-que-e-o-docker}
  \end{alineas}

  \item Kubernetes

  \begin{alineas}
     \item Para proporcionar escalabilidade horizontal de maneira automatizada através de métricas
           - como alto uso de CPU \cite{kubernetes-horizontal-pod-autoscaling}
  \end{alineas}

\end{alineas}

\section{Organização}

O trabalho é divido em 4 partes:

\begin{itemize}
    \item \autoref{introducao} Introdução: nesta primeira parte é introduzido o que
          será tratado no trabalho, explicando o problema e deixando claros os
          objetivos que deverão ser atendidos ao final;

    \item \autoref{conceitos} Conceitos: nesta parte são discutidos os fundamentos e
          argumentos para as tomadas de decisões de quais serão as melhores
          abordagens para resolver o problema;

    \item \autoref{implementacao} Implementação: nesta parte são explicadas as
          implementações do código fonte do serviço de reserva de ingressos, assim
          como, de como foi implementada e implantada a infraestrutura;

    \item \autoref{resultados} Resultados: nesta parte são exibidos os resultados
          práticos da implementação e conclusão das escolhas tomadas.

\end{itemize}
