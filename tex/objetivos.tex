\chapter{Objetivo}

O objetivo principal é tornar viável a arquitetura de um sistema web, atendendo com eficiência
as oscilações das quantidades de requisições que recebe, de altas demandas aos tempos de ociosidade.
Por exemplo, no momento da abertura das vendas de ingressos para um determinado evento
que ocorre normalmente em datas e horários específicos. É necessário atender grande demanda
em momentos de venda e também é preciso que haja otimização do sistema quando estiver ocioso.
Ou seja, reduzir a quantidade de recursos utilizados quando não houver
grande demanda - reduzindo o custo.

\section{Escalabilidade e microsserviços}
A escalabilidade (\autoref{escalabilidade}) é a parte mais relevante desse projeto.
Tornar um sistema escalável é um grande desafio.
Para ser possível, vários pontos da arquitetura devem ter a
abordagem adequada. Por exemplo, a forma como o sistema é dividido implica em vários
fatores.
A abordagem de microsserviços (\autoref{microservicos}) é uma técnica que será abordada
com o intuito de evitar pontos únicos de falha e boa divisão de responsabilidades
- através de contextos bem definidos, possibilitando a criação de serviços que
atendam especificamente um contexto.
A utilização do conceito de microsserviços possibilitará a escalabilidade de maneira
simplificada, rápida e de acordo com o que é visto atualmente no mercado.

\section{Implementação da reserva de ingressos}

A parte prática será desenvolvida apenas em relação ao serviço de backend que
atenderá requisições no contexto de reserva de ingressos.
Porém, um sistema de vendas de ingressos online é abrangente e envolve divulgação,
exibição de eventos, assentos disponíveis, valores e reserva do ingresso -
que é limitada, assim como a venda e suporte pós-venda. Um sistema completo possui essas
e outras características.

A implementação do serviço de reserva de ingressos terá o código discutido
em detalhes (\autoref{implementacao-do-servico}). Não há implementação
de interface do usuário (frontend), apenas a implementação dos algoritmos
que rodam nos servidores (backend).

Com microsserviços, vários dos itens abordados poderão ser aplicados em outras partes
de um sistema
de venda de ingressos, em formato de serviços que terão os mais diversos contextos.
A aplicação terá uma API REST (\autoref{api}) que responderá requisições
seguindo o protocolo HTTP.

\section{Infraestrutura para rodar o serviço}

O serviço será uma aplicação que rodará em um ambiente Linux através
de uma imagem Docker e poderá ser replicada com Kubernetes.
Estas ferramentas serão abordadas, assim como, técnicas de programação
voltadas a serviços de alto desempenho.

A sugestão é que essa infraestrutura inteira funcione em alguma plataforma em nuvem
(\autoref{infraestrutura-em-nuvem}) para que se obtenham os recursos e seja possível atingir
a escalabilidade necessária no momento de alta demanda da venda de ingressos.

\subsection{Ferramentas de infraestrutura}

\index{alíneas}\index{subalíneas}\index{incisos}Ter vários microsserviços implica em gerenciar
vários sistemas.
Para ter sucesso neste gerenciamento é fundamental que haja automação
\cite{martin-fowler-microservices}.
Para simplificar o gerenciamento dos serviços serão mencionadas algumas ferramentas que
irão facilitar a manutenção, escalabilidade e monitoramento:

\begin{alineas}

  \item Docker (\autoref{docker})

  \begin{alineas}
     \item Para criação, teste e implantação de aplicações \cite{aws-o-que-e-o-docker}
     \item Os scripts para rodar a aplicação usaram a ferramenta
           (\autoref{como-rodar-a-aplicacao})
  \end{alineas}

  \item Kubernetes (\autoref{kubernetes})

  \begin{alineas}
     \item Para proporcionar escalabilidade horizontal de maneira automatizada através de métricas
           - como alto uso de CPU \cite{kubernetes-horizontal-pod-autoscaling}
  \end{alineas}

\end{alineas}
