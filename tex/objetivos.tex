\chapter{Objetivo}

O objetivo principal é tornar viável a arquitetura de um sistema web de atender
alta demanda quando solicitado. Por exemplo, no momento da abertura das vendas
de ingressos para um determinado evento. Também é necessário que haja otimização
do sistema quando o sistema estiver ocioso. Ou seja, reduzir a quantidade de
recursos utilizados quando não houver grande demanda - reduzindo o custo.

\section{Microserviços}

A escalabildiade é a parte mais relevante desse projeto. Tornar um sistema escalável
é um grande desafio. Para ser possível, vários pontos da arquitetura devem ter a
abordagem adequada. Por exemplo, a forma como o sistema é divido implica em vários
fatores. A abordagem de microserviços evita que haja um ponto único de falha.

Com microserviços é possível escalar apenas algumas partes do sistema, diminuindo
o custo e esforço para aumentar o poder de processamento para atender muitas requisições.

Pode haver um serviço que cuida do contexto do usuário; outro serviço que lida com a
lista de eventos disponíveis; outro serviço responsável por realizar reserva de vaga de
cada evento; e assim por diante. Ou seja, cada um desses serviços pode ter a sua a quantidade
de requisições diferentes umas das outras, mas os serviços poderão ser escalados de forma
independente e conforme a necessidade.

Um exemplo real pode ser objservado em um e-commerce.

\begin{citacao}

Ao criarmos a Editora Casa do Código, decidimos seguir outro caminho para o e-commerce.
Lá não podemos correr o risco de ficar com a loja fora do ar caso alguma funcionalidade
periférica falhe, então quebramos nossa arquitetura em serviços menores. Dessa forma, há
um sistema principal, que é a loja e está hospedada no Shopify, e vários outros sistemas
que gravitam em torno dela. Temos uma aplicação que faz a liberação dos e-books para os
clientes, outro que contabiliza os royalties para autores, outro para cuidar da logística
de envio dos livros impressos para o cliente, um painel de visualização dos livros comprados,
um para fazer a liberação de vale presentes e outro para promoções.

Dessa forma, quando um evento acontece na nossa loja online, os diferentes sistemas precisam
ser notificados. Para essas notificações usamos requisições HTTP, assim, quando uma compra
é confirmada na loja online, todos os sistemas recebem em um Endpoint essa requisição HTTP,
contendo um JSON com todos os dados da compra. Cada sistema decide o que fazer com as informações
recebidas, de acordo com a necessidade. Por exemplo, o sistema de liberação dos e-books verifica
se a compra possuía e-books e gera os links de download para o comprador, enquanto que o sistema
de logística já dá baixa no estoque quando a compra é de um impresso, além de notificar as pessoas
quando algum livro está ficando com estoque crítico. Essa característica importante para o maior
desacoplamento dos serviços é conhecida como Smart endpoints, dumb pipes, que em uma tradução
livre pode ser entendida como Endpoints inteligentes e fluxos
simples. \cite{arquitetura-de-microservicos-ou-monolitica}.

\end{citacao}

\section{Gerenciamento de serviços}


\index{alíneas}\index{subalíneas}\index{incisos}Ter vários microserviços implica em gerenciar
vários sistemas. E isso significa que automação é fundamental para que isso funcione
\cite{martin-fowler-microservices}.
Para simplificar o gerenciamento dos serviços serão utilizadas algumas ferramentas que
irão facilitar a manutenção, escalabildiade e monitoramento:

\begin{alineas}

  \item Docker

  \begin{alineas}
     \item Para criação, teste e implantação de aplicações rapidamente \cite{aws-o-que-e-o-docker}
  \end{alineas}

  \item Kubernetes

\end{alineas}
