\chapter{Objetivo}

O objetivo principal é tornar viável a arquitetura de um sistema web de atender
alta demanda quando solicitado. Por exemplo, no momento da abertura das vendas
de ingressos para um determinado evento. Também é necessário que haja otimização
do sistema quando o sistema estiver ocioso. Ou seja, reduzir a quantidade de
recursos utilizados quando não houver grande demanda - reduzindo o custo.

\section{Microserviços}

A escalabildiade é a parte mais relevante desse projeto. Tornar um sistema escalável
é um grande desafio. Para ser possível, vários pontos da arquitetura devem ter a
abordagem adequada. Por exemplo, a forma como o sistema é divido implica em vários
fatores. A abordagem de microserviços evita que haja um ponto único de falha. A
utilização do conceito de microserviços possibilitará a escalabilidade de maneira
simplificada, rápida e de acordo com o que é visto atualmente no mercado.

\section{Gerenciamento dos serviços}

\index{alíneas}\index{subalíneas}\index{incisos}Ter vários microserviços implica em gerenciar
vários sistemas. E isso significa que automação é fundamental para que isso funcione
\cite{martin-fowler-microservices}.
Para simplificar o gerenciamento dos serviços serão utilizadas algumas ferramentas que
irão facilitar a manutenção, escalabildiade e monitoramento:

\begin{alineas}

  \item Docker

  \begin{alineas}
     \item Para criação, teste e implantação de aplicações rapidamente \cite{aws-o-que-e-o-docker}
  \end{alineas}

  \item Kubernetes

  \begin{alineas}
     \item Para proporcionar escalabildiade horizontal de maneira automatizada através de métricas
           - como alto uso de CPU \cite{kubernetes-horizontal-pod-autoscaling}
  \end{alineas}

\end{alineas}
