\chapter{Objetivo}

O objetivo principal é tornar viável a arquitetura de um sistema web de atender
alta demanda quando solicitado. Por exemplo, no momento da abertura das vendas
de ingressos para um determinado evento. Também é necessário que haja otimização
do sistema quando o sistema estiver ocioso. Ou seja, reduzir a quantidade de
recursos utilizados quando não houver grande demanda - reduzindo o custo.

\section{Microserviços}

A escalabildiade é a parte mais relevante desse projeto. Tornar um sistema escalável
é um grande desafio. Para ser possível, vários pontos da arquitetura devem ter a
abordagem adequada. Por exemplo, a forma como o sistema é divido implica em vários
fatores. A abordagem de microserviços evita que haja um ponto único de falha. A
utilização do conceito de microserviços possibilitará a escalabilidade de maneira
simplificada, rápida e de acordo com o que é visto atualmente no mercado.

\section{Gerenciamento dos serviços}

\index{alíneas}\index{subalíneas}\index{incisos}Ter vários microserviços implica em gerenciar
vários sistemas. E isso significa que automação é fundamental para que isso funcione
\cite{martin-fowler-microservices}.
Para simplificar o gerenciamento dos serviços serão utilizadas algumas ferramentas que
irão facilitar a manutenção, escalabildiade e monitoramento:

\begin{alineas}

  \item Docker

  \begin{alineas}
     \item Para criação, teste e implantação de aplicações rapidamente \cite{aws-o-que-e-o-docker}
  \end{alineas}

  \item Kubernetes

  \begin{alineas}
     \item Para proporcionar escalabildiade horizontal de maneira automatizada através de métricas
           - como alto uso de CPU \cite{kubernetes-horizontal-pod-autoscaling}
  \end{alineas}

\end{alineas}

Um sistema de vendas de ingressos online é abrangente e vai da parte de divulgação,
exibição de eventos, lugares e valores disponíveis, passando pela reserva do ingresso,
que é limitada, assim como a venda e suporte pós-venda. Um sistema completo possui essas
e outras características. Do ponto de visto técnico, várias das abordagens discutidas
podem ser utilizadas em muitos desses pontos. Porém, a parte prática será desenvolvida
apenas em relação ao serviço de backend que atentererá requisições do contexto de reserva
de ingressos.

O serviço será uma aplicação que rodará de forma facilitada em um ambiente Linux através
de uma imagem Docker e poderá ser replicada com Kubernetes. Todas as essas ferramentas
serão abordadas, assim como, técnicas de programação voltadas a seriços de alto
desempenho. A aplicação terá uma API REST que responderá requisições HTTP.

A sugestão é que essa infraestrutura inteira rode em alguma plataforma em nuvem para
que se obtenham os recursos necessários para que seja possível obter alta escalabilidade,
performance e disponibilidade necessária no momento da alta demanda da reserva de ingressos.
