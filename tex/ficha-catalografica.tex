% Isto é um exemplo de Ficha Catalográfica, ou ``Dados internacionais de
% catalogação-na-publicação''. Você pode utilizar este modelo como referência.
% Porém, provavelmente a biblioteca da sua universidade lhe fornecerá um PDF
% com a ficha catalográfica definitiva após a defesa do trabalho. Quando estiver
% com o documento, salve-o como PDF no diretório do seu projeto e substitua todo
% o conteúdo de implementação deste arquivo pelo comando abaixo:
%
% \begin{fichacatalografica}
%     \includepdf{fig_ficha_catalografica.pdf}
% \end{fichacatalografica}
\begin{fichacatalografica}
	\vspace*{\fill}					% Posição vertical
	\hrule							% Linha horizontal
	\begin{center}					% Minipage Centralizado
	\begin{minipage}[c]{12.5cm}		% Largura

	Formento, André

    \hspace{-1.4cm} F725s
    \hspace{0.79cm} \imprimirtitulo  / \imprimirautor~--~2017

    \hspace{0.5cm} \cfoot{{\hypersetup{linkcolor=black}\pageref{LastPage}}}fls.; il.; color.\\

	\hspace{0.5cm} \imprimirorientadorRotulo~Prof. \imprimirorientador\\

	\hspace{0.5cm} \parbox[t]{\textwidth}{\imprimirtipotrabalho~--~\imprimirinstituicao, 2017.}\\

	\hspace{0.5cm}
		1. Escalabilidade
		2. Performance
		3. Disponibilidade
		4. Microsserviço
		I. Fonseca, José A.
		II. \imprimirtitulo.\\

	\hspace{8.75cm} CDU 004.42\\

	\end{minipage}
	\end{center}
	\hrule
\end{fichacatalografica}
% ---
