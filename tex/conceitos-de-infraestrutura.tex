\chapter{Conceitos de infraestrutura}

Para rodar um software é necessário que haja um hardware. De modo geral,
há duas estratégias de infraestrutura em que o hardware (servidor) antenderá a aplicação (software):
local (\autoref{infraestrutura-local}) ou em nuvem (\autoref{infraestrutura-em-nuvem}).

\begin{citacao}

Quando se trata de determinar a melhor forma de armazenar enormes quantidades de
dados empresariais confidenciais gerados diariamente, as organizações se deparam
com duas opções: (aquilo que pode ser considerado o método tradicional) sistemas
de armazenamento nas próprias instalações ou uma solução externa hospedada por
provedores de computação na nuvem. Embora muitas empresas continuem investindo
em armazenamento local, o armazenamento baseado na nuvem está começando a se
tornar uma opção potencial para algumas. De fato, espera-se que 36\% de todos
os dados sejam armazenados na nuvem até 2016, um crescimento da parcela de
apenas 7\% em 2013. Obviamente, apesar de o armazenamento em nuvem parecer
uma opção intrigante, existem pontos positivos e negativos associados a cada
método, desde custo e controle até segurança \cite{armazenamento-no-local-ou-na-nuvem}.

\end{citacao}

\section{Infraestrutura local}\label{infraestrutura-local}

Quando a insfraestrutura é local toda a responsabilidade é da empresa - manutenção,
compra de equipamentos e replicação do ambiente.
Ou seja, há necessidade atualizar software, comprar licenças de todos os programas,
fazer backup, garantir segurança física e lógica.
Todos os detalhes de como manter essa infra também são de responsabilidade da própria empresa -
como contratação de pessoas qualificadas e políticas de atualização.

Conforme \cite{beneficios-da-computacao-em-nuvem-para-sua-startup}, várias startups
tem optado por computação em nuvem por diversos motivos. Dentre eles, um que se destaca
em algo que pode ser muito relevante para a solução do problema da venda de ingressos, 
a escalablidade.

\begin{citacao}
Escalabilidade: Algo que é fundamental em qualquer startup é ser escalável.
Com infraestrutura em nuvem, sua empresa tem flexibilidade para aumentar e
diminuir os recursos conforme a demanda, sem a necessidade de comprar novos equipamentos, 
instalações de sistemas ou configurações complexas 
\cite{beneficios-da-computacao-em-nuvem-para-sua-startup}.

\end{citacao}

Em uma organização que possui uma variação muito grande da quantidade de demanda, lidar
com a escalablidade pode se tornar algo desastroso quando fala-se em custos. Imaginando
que a infraestrutura para atender a demanda da venda de ingressos estará limitada
pelos recursos de hardware que a organização terá, é possível imaginar que isso trará
vários problemas.

Por exemplo, Um cenário onde em um determinado horário a demanda por compra de ingressos 
esteja atingindo altos indices, a ponto de a infraestrutura alocada não suporte atender 
todas as requisições. Em um primerio momento, enxerga-se como solução a aquisição de mais 
equipemantos para suprir essa ineficiência. Porém, fora deste período de muitas vendas, essa
infraestrutura seria subutilizada. Em um caso pior, pode ser que haja mais de um evento que
tenham os ingressos com início das vendas no mesmo período. Desta forma, fica claro o problema
de tentar manter um infraestrutura própria para atender todas as demandas.

A ideia principal é que essa infraestrutura seja obtida de forma dinâmica. Por isso,
ter uma infraestrutura local, ou própria, é inviável para a solução do problema da
venda de ingressos.

\section{Infraestrutura em nuvem}\label{infraestrutura-em-nuvem}

Conforme \cite{what-is-cloud-computing}, computação em nuvem proporciona poder computacional
sob demanda através da internet. Desta forma, é possível que haja aumento de poder
de processamento conforme a necessidade do negócio - pagando apenas pelo que for utilizado.

\begin{citacao}
A computação em "nuvem", o chamado cloud computing, é um conceito que permite o acesso a
um conjunto de recursos como de processamento, armazenamento, conectividade, plataformas,
aplicações e serviços disponibilizados na Internet. Conjuntos de supercomputadores, que operam em
rede, formam a chamada "nuvem", que armazena softwares, documentos e aplicativos de um sistema.
Para ter acesso aos arquivos, basta que o usuário do cloud tenha acesso à internet
\cite{cloud-computing-conceitos-e-perspectivas-2012}.
\end{citacao}

Não é necessário grande investimento inicial para disponibilizar uma aplicação web.
Como o pagamento é pelo que for utilizado, se uma aplicação em determinado período tem
poucos acessos, então, pagará pouco; nos momentos onde há grande demanda, pagará por aquilo
que estiver utilizando. E é justamente essa flexibilidade que é fundamental para o sucesso
da solução proposta. Ou seja, toda a demanda de compra de ingressos poderá ser atendida
sem necessidade de investimentos em infraestrutura.

Conforme \cite{types-of-cloud-computing}, a computação em nuvem facilita a disponibilização
de aplicações aos desenvolvedores fazendo com quem foquem seu trabalho no desenvolvimento
do software.
Desta forma, ter toda a infraestrutura em nuvem é uma ótima escolha para lidar um sistema
de reserva de ingressos.

% pelo que vi tem mais artigos aqui: \url{https://canaltech.com.br/corporate/computacao-na-nuvem/}
% http://www.each.usp.br/dc/aulas/ComputacaoEmNuvem-DanielCordeiro.pdf

\subsection{Tipologia das plataformas em nuvem}\label{tipologia-das-plataformas-em-nuvem}

Com a popularização da computação em nuvem houve um aumento dos produtos e opções de
serviços disponíveis.
Desta forma, foram criados vários modelos e estratégias para utilização da computação
em nuvem, onde cada uma dessas estratégias foca em um determinado nível de abstração.
Há três principais formatos de nuvem que podem ser explorados:
IaaS (\autoref{iaas}), PaaS (\autoref{paas}) e SaaS (\autoref{saas}).
Recentamente vem surgindo um quarto formato que ainda é visto pelo mercado com
cautela: BaaS (\autoref{baas}).

% https://sistemas.eel.usp.br/docentes/arquivos/5840003/444/Cloudcomputing.pdf
% pagina 10

%\url{https://aws.amazon.com/pt/types-of-cloud-computing/}


\subsubsection{IaaS}\label{iaas}

No modelo Infrastructure as a Service (IaaS), ou Infra-estrutura como Serviço,
não há controle sobre o sistema operacional. O objetivo é o fornecimento
de recursos de computação necessários para rodar uma aplicação. Há acesso (virtual
ou hardware dedicado) em recursos de rede, computadores e acesso para armazenamento
de dados.
Um exemplo é o produto EC2 da Amazon.

%pode especificar cada uma delas:
%\url{https://apprenda.com/library/paas/iaas-paas-saas-explained-compared/}
%\url{https://antonioricardo.org/2013/03/28/o-que-e-saas-iaas-e-paas-em-cloud-computing-conceitos-basicos/}

\subsubsection{PaaS}\label{paas}

No modelo Plataform as a Service (PaaS), ou Plataforma como Serviço,
o foco é no gerenciamento da implantação e das aplicações.
Não há preocupação com aquisição de recursos, planejamento de capacidade,
manutenção de software, patching ou qualquer outro trabalho de baixo nível.
O Google AppEngine e a Amazon S3 são exemplos de PaaS.

\subsubsection{SaaS}\label{saas}

No modelo Software as a Service (SaaS), ou Software como Serviço,
há um produto completo, podendo ser oferecido inclusive como um aplicação
para o usuário final.
Aqui não há preocupações com serviço e nem gerenciamento de infraestrutura.
Um exemplo claro é o serviço de webmail, assim como Facebook,
Google Docs e Linkedin.

\subsubsection{BaaS}\label{baas}

Para esse modelo que é mais novo, há várias formas de definí-lo:
Backend as a Service (BaaS) ou Software como Serviço; Serverless Architectures; ou ainda,
Functions as a service (FaaS).
Nele fica abstraído inclusive a arquitetura de software da aplicação,
tendo como opção apenas desenvolver funções internas para execução. Desta forma, há grande
dependência da plataforma em que roda a aplicação.
Há vários serviços que podem ser facilmente consumidos como, serviço de chat, segurança com
OAuth, dentre outros.
Conforme \cite{martin-serverless}, há uma dependência também do ambiente em que roda o
código fonte, podendo inclusive necessitar de manutenção posterior em algo que está
funcionando pelo simples fato do fornecedor realizar alguma atualização.
Ou seja, há uma forte dependência com o fornecedor.

Alguns exemplos de BaaS são a AWS Lambda e Google Firebase.

\begin{citacao}
Um BaaS pode ser visto como uma ponte conectando o backend e o frontend de uma aplicação.
Os BaaS auxiliam os desenvolvedores a acelerar a criação de aplicações web e mobile e
simplificam a criação de APIs. Em vez de codificar o backend inteiro, o desenvolvedor usa
o BaaS para criar as APIs e conectá-las às aplicações \cite{backend-as-a-service-pros-e-contras}.
\end{citacao}

\subsubsection{Escolha da tipologia}

Cada tipologia discutida possui um nível de abstração que é proporcional ao nível de
dependência da plataforma, assim como, menor opção de customização para atender
o próprio negócio. Além do que, se for necessário um seriço específico, provavelmente
não existirá. Como é o caso da reserva de ingressos.

Por isso, a escolha da tipologia foi de utilizar o modelo IaaS \autoref{iaas},
onde será possível instalar e escalar o serviço desenvolvido.
Essa infraestrutura em nuvem não foi implementada, mas da maneira como foi construído
e disponibilizado o serviço de reserva de ingressos, é possível usar qualquer modelo
de de IaaS disponível hoje no mercado, dependendo apenas de um sistema operacional
Linux.

% está é uma subsection de infraestrutura em nuvem do arquivo infraestrutura.tex

\subsection{Plataformas em nuvem}

Há diversas opções de plataformas em nuvem no mercado: AWS, Google Cloud Plataform,
Openstack Cloud, Microsoft Azure, Digital Ocean, etc. Dentre essas, há diferenças
de produtos oferecidos, preços, dentre outros fatores. Não será discutido aqui
especificidades de cada um.

Para o problema da reserva de ingressos, poderia ser usado, por exemplo, a AWS
que proporciona autoscaling \cite{aws-autoscaling} juntamente com o
Kubernetes \autoref{kubernetes}.
Desta forma, toda a infraestrutura seria escalável numa plataforma em nuvem
que hoje é considerada uma das maiores do mundo.

% \subsubsection{@AWS}
%
% Amazon Web Services (AWS)
%
% \url{https://aws.amazon.com/pt/}
%
% \url{https://aws.amazon.com/pt/blogs/compute/kubernetes-clusters-aws-kops/}
%
% \url{https://aws.amazon.com/pt/what-is-aws/}
%
% \url{https://aws.amazon.com/pt/about-aws/}
%
% tente focar no que ela pretende resolver (infraestrutura) e principalmente
% fazer links com o nosso tcc
%
% \subsubsubsection{@Auto Scaling na AWS}
% pode falar de produtos como Auto Scaling
% \url{https://aws.amazon.com/pt/autoscaling}
%
% e procura e me pergunta sobre outros assuntos relevantes pro tcc
%
%
%
% \subsubsection{@Google Cloud Platform (GCP)}
%
% nesse link também tem outros assuntos relacionados do lado direito
% \url{https://cloud.google.com/docs/overview/?hl=pt-br}
%
% você pode focar mais na AWS e não tanto no google plataform (se julgar necessario)
%
%
% \url{http://selbielabs.com/cloud-platforms-compared/}
%
% \subsubsection{@Heroku}
%
% \url{https://devcenter.heroku.com/}
%
% \subsubsection{@Digital Ocean}
%
% \subsubsection{@Microsoft Azure}
%

