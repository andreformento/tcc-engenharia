\chapter{Arquitetura}\label{arquitetura}

Na figura ~\ref{fluxo-de-reserva-de-ingressos-da-aplicacao}
é exibido o fluxo de como ocorre o consumo do serviço de reserva de ingresso.

\begin{figure}[!htp]
  \centering
  \scalebox{1}{
    \begin{sequencediagram}
        \newthread{cliente}{\shortstack{Cliente \\ \\ \\
          \begin{tikzpicture}
            \node [copy shadow,fill=gray!20,draw=black,thick ,align=center] {Mobile \\ Website \\ etc};
          \end{tikzpicture}}
        }{}
        \newinst[1]{api-gateway}{\shortstack{API Gateway \\ \\
          \begin{tikzpicture}
            \node [copy shadow,fill=gray!20,draw=black,thick ,align=center] {Balanceador \\ de cargas};
          \end{tikzpicture}}
        }{}
        \newinst[1]{servico-de-reservas}{\shortstack{Serviço de reservas \\ \\
          \begin{tikzpicture}
            \node [copy shadow,fill=gray!20,draw=black,thick ,align=center] {Aplicação Java};
          \end{tikzpicture}
        }}{}
        \newinst[1]{banco-de-dados}{\shortstack{Banco de dados \\
          \begin{tikzpicture}[shape aspect=.5]
            \tikzset{every node/.style={cylinder, shape border rotate=90, draw,fill=gray!25}}
            \node at (1.5,0) {Redis};
          \end{tikzpicture}
        }}{}

        \begin{call}{cliente}{}{api-gateway}{}
          \begin{call}{api-gateway}{}{servico-de-reservas}{}
            \begin{call}{servico-de-reservas}{}{banco-de-dados}{} \end{call}
          \end{call}
        \end{call}
    \end{sequencediagram}
  }
  \caption{Fluxo de reserva de ingressos da aplicação}
  \label{fluxo-de-reserva-de-ingressos-da-aplicacao}
\end{figure}

No contexto da aplicação, cliente web é definido como qualquer consumidor
de API REST (\autoref{rest}) que utilize os dados de um serviço.
O cliente web pode ser um site web, um aplicativo mobile ou até mesmo um serviço
externo que se integra ao sistema de venda de ingressos.

O cliente web conhecerá apenas o endereço do API Gateway (\autoref{api-gateway})
que será responsável por fazer o balanceamento de cargas
(\autoref{balanceador-de-carga}) do serviço de reserva de ingressos.
Toda a regra de negócios refente a reserva de ingressos é executada no
serviço correspondente que possui acesso a um banco de dados (\autoref{banco-de-dados}).
Este responde ao API Gateway que retorna o resultado ao cliente.

Fica abstraído como o cliente é exibido ao usuário
final, possibilitando assim que vários tipos diferentes de clientes implementem
a sua forma de exibição e a própria experiência de visualização para o usuário.
Ou seja, a arquitetura proporciona a clara divisão entre backend
(servidores da aplicação) com frontend (site web, mobile, etc).

Desta arquitetura proposta, a parte prática do trabalho ocorreu apenas na
implementação do serviço de reserva de ingressos (\autoref{implementacao-do-servico}),
deixando o restante apenas como trabalhos futuros (\autoref{trabalhos-futuros}).
