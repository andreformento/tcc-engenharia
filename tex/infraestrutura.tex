\chapter{Infraestrutura}

Para rodar um software é necessário que haja um hardware. Para isso, é necessário
que haja um servidor (hardware) para rodar uma aplicação (software). De modo geral,
há duas estratégias de infraestrutura:
local (\autoref{infraestrutura-local}) ou em nuvem (\autoref{infraestrutura-em-nuvem}).

\begin{citacao}

Quando se trata de determinar a melhor forma de armazenar enormes quantidades de
dados empresariais confidenciais gerados diariamente, as organizações se deparam
com duas opções: (aquilo que pode ser considerado o método tradicional) sistemas
de armazenamento nas próprias instalações ou uma solução externa hospedada por
provedores de computação na nuvem. Embora muitas empresas continuem investindo
em armazenamento local, o armazenamento baseado na nuvem está começando a se
tornar uma opção potencial para algumas. De fato, espera-se que 36\% de todos
os dados sejam armazenados na nuvem até 2016, um crescimento da parcela de
apenas 7\% em 2013. Obviamente, apesar de o armazenamento em nuvem parecer
uma opção intrigante, existem pontos positivos e negativos associados a cada
método, desde custo e controle até segurança \cite{armazenamento-no-local-ou-na-nuvem}.

\end{citacao}

\section{Infraestrutura local}\label{infraestrutura-local}

Quando a insfraestrutura é local todas as responsabilidades são de à possui.
Então há necessidade dar manutenção,
atualizar software, comprar licenças de todos os programas, fazer backup, garantir
segurança física e lógica, etc. Todos detalhes de como manter essa infra
também são de responsabilidade própria - como contratação de pessoas qualificadas,
políticas de atualização, segurança, e assim por diante.

Conforme \cite{beneficios-da-computacao-em-nuvem-para-sua-startup}, várias startups
tem optado por computação em nuvem por diversos motivos. Dentre eles, um que se destaca
em algo que pode ser muito relevante para a solução do problema da venda de ingressos é
a escalablidade.

\begin{citacao}
Escalabilidade: Algo que é fundamental em qualquer startup é ser escalável.
Com infraestrutura em nuvem, sua empresa tem flexibilidade para aumentar e
diminuir os recursos conforme a demanda sem comprar novos equipamentos ou
configurações complexas e instalações de
sistemas \cite{beneficios-da-computacao-em-nuvem-para-sua-startup}.

\end{citacao}

Em uma organização que possui uma variação muito grande da quantidade demanda, lidar
com a escalablidade pode se tornar desastroso quando for lidar com custos. Imaginando
que a infraestrutura para atender a demanda da venda de ingressos estará limitada
pelos recursos de hardware que a organização terá é possível imaginar que isso trará
vários problemas. Por exemplo, quando a demanda for alta para vender ingressos em um
determinado horário porque em alguma data é iniciada a venda de ingressos para um evento
que tem muitos ingressos, precisaria de muita infraestrutura para atender a todas as
requisições de modo que não houvesse lentidão ou indisponibilidade. Porém, fora deste
período de muitas vendas essa infraestrutura seria subutilizada. Em um caso pior, pode
ser que haja mais de um evento que tenham os ingressos com início das vendas no mesmo
período. Desta forma, fica claro o problema de tentar manter um infraestrutura própria
para atender todas as demandas.

A ideia principal é que essa infraestrutura seja obtida de forma dinâmica. Por isso,
ter uma infraestrutura local, ou própria, é inviável para a solução do problema da
venda de ingressos.

\section{@Infraestrutura em nuvem}\label{infraestrutura-em-nuvem}

\url{https://sistemas.eel.usp.br/docentes/arquivos/5840003/444/Cloudcomputing.pdf}

% https://aws.amazon.com/pt/what-is-cloud-computing/

pelo que vi tem mais artigos aqui: \url{https://canaltech.com.br/corporate/computacao-na-nuvem/}

\url{https://aws.amazon.com/pt/types-of-cloud-computing/}

\subsection{@IaaS}

pode especificar cada uma delas:
\url{https://apprenda.com/library/paas/iaas-paas-saas-explained-compared/}

\url{https://antonioricardo.org/2013/03/28/o-que-e-saas-iaas-e-paas-em-cloud-computing-conceitos-basicos/}

\subsection{@PaaS}

\subsection{@SaaS}
