\chapter{Implementação do serviço}\label{implementacao-do-servico}

% \section{@@arrumar essa parte do código fonte}
% [nao é pra comecar falando de testes. primeiro explicar o que foi implementado,
% ferramentas, o que é esse serviço e por fim mencionar testes, ou nao]

A parte prática do trabalho implementa o serviço de reserva de ingressos.
Para reservar um ingresso é necessário que haja a informação de quantos ingressos
estão disponíveis em um determinado evento. Por questão de escopo do trabalho,
foram ignorados detalhes como
tipo de ingresso por setor, para estudantes, idosos e qualquer outra especificidade
de negócio. Mas, isso seria resolvido com pequenas alterações na aplicação
e não impacta na solução desenvolvida.

O serviço possui código fonte open-source e pode ser acessado em
\url{https://github.com/andreformento/term-paper}.

\section{Código fonte}

Este serviço é uma aplicação escrita na linguagem Java na versão 8. O paradigma
da implementação do código fonte foi Orientação à Objetos. Porém, possui alguns
princípios de programação funcional - como é o caso da técnica de imutabilidade.

Detalhes de requisitos (\autoref{requisitos-da-aplicacao}) e a
forma como rodar o programa (\autoref{como-rodar-a-aplicacao}) serão detalhadas
na parte de implementação da infraestrutura (\autoref{implementação-da-infraestrutura}).

\index{alíneas}\index{subalíneas}\index{incisos}Conforme
\cite{does-immutability-really-mean-thread-safety} e \cite{java-doc-immutable-objects},
a técnica de imutabilidade traz vários benefícios:

\begin{alineas}

  \item Objetos imutáveis facilitam a programação concorrente;

  \item Como a implementação foi em Java, traz benefícios quanto ao algorítmo do
        Garbage Collector;

  \item Mudança indesejada em códigos dispersos;

  \item Simplicidade de manutenção do código.

\end{alineas}

\lstset{language=Java,keywordstyle={\bfseries \color{blue}}}

O primeiro ponto é o mais relevante para a solução, visto que, concorrência
eficiente é um dos pontos mais importantes. Com a palavra chave \textbf{final} é
possível definir que os atributos da classe não poderão ser alterados depois
de inicializados, conforme exemplo do \autoref{classe-ticket-reservation}. O
segundo item também influencia no desempenho da aplicação.

\begin{lstlisting}[label=classe-ticket-reservation,caption=Classe TicketReservation em Java]
public class TicketReservation implements Serializable {

    @NotNull
    private final String idEvent;

    @NotNull
    private final String idUser;

    @ConstructorProperties({"idEvent", "idUser"})
    public TicketReservation(String idEvent, String idUser) {
        this.idEvent = idEvent;
        this.idUser = idUser;
    }
    // getters, setters, equals...
}
\end{lstlisting}

\subsection{Spring Boot Framework}

Foi utilizado um framework em Java que possibilita a rápida criação de aplicações.
Conforme \cite[8]{spring-boot-reference-guide}, o Spring Boot Framework torna
produtiva a criação de serviços que rodam de maneira independente, isolada
e com várias opções de acoplar outras bibliotecas que são comuns no desenvolvimento
de software, como integração com banco de dados, sistemas de mensageria, geração
de documentação a partir do código, etc.

\subsection{API}\label{api}

A aplicação desenvolvida gera uma API REST (\autoref{rest}) que é responsável por fazer uma reserva
no evento solicitado. Desta forma, este serviço atende o escopo de reserva de lugar no
evento. Conforme exibido no \autoref{classe-ticket-reservation-reservation-controller}, existe um endpoint responsável
por fazer essa reserva. Ele tem um contrato que permite que seja informado o evento e
o usuário para realizar a reserva, simplificando a sua utilização.

\index{alíneas}\index{subalíneas}\index{incisos}
Conforme descrito por \cite{mark-richards-software-architecture-patterns},
foi utilizada a arquitetura por camadas para organizar as classes.
A classe \textbf{TicketReservationController} está na camada \textbf{presentation} e tem a
responsabilidade de receber requisições, converter para o padrão utilizado na
aplicação e chamar a classe responsável pela lógica.
Esta classe possui dependência de duas classes:

\begin{alineas}

  \item \textbf{TicketReservationService}

  \begin{alineas}
     \item Fica na camada \textbf{business};
     \item Responsável por lidar com as regras de negócio/lógica do ingresso;
     \item É ela a responsável por lidar com as validações;
  \end{alineas}

  \item \textbf{TicketReservationMapper}

  \begin{alineas}
     \item Fica na camada \textbf{presentation};
     \item Responsável por converter o que vem de fora da aplicação e transformar
           no padrão adotado dentro da aplicação;
     \item E também o inverso do item anterior: o que for gerado dentro da aplicação -
           seguindo um padrão definido, ela é responsável por converter
           os dados a serem expostos de modo que mantenha um contrato pré-determinado
  \end{alineas}

\end{alineas}

\begin{lstlisting}[label=classe-ticket-reservation-reservation-controller,caption=Classe TicketReservationController em Java]
@RestController
@RequestMapping("/events")
public class TicketReservationController {

    private final TicketReservationService service;
    private final TicketReservationMapper mapper;

    public TicketReservationController(
          TicketReservationService service,
          TicketReservationMapper mapper) {
        this.service = service;
        this.mapper = mapper;
    }

    @PostMapping
    @RequestMapping("/{idEvent}/tickets")
    public HttpEntity<Resource<TicketReservationResponse>> booking(
        @PathVariable("idEvent") String idEvent,
        @RequestBody final
             TicketReservationRequest TicketReservationRequest
    ) {

        return mapper.mapToResponse(service.booking(mapper.
            mapFromRequest(idEvent, TicketReservationRequest)));
    }

}
\end{lstlisting}

\subsection{REST}\label{rest}

\begin{citacao}

REST é um dos modelos de arquitetura que foi descrito por
Roy Fielding \cite[5]{rest-roy-thomas-fielding},
um dos principais criadores do protocolo HTTP, em sua tese de doutorado
e que foi adotado como o modelo a ser utilizado na evolução da arquitetura
do protocolo HTTP.

[...] REST na verdade pode ser considerado como um conjunto de princípios,
que quando aplicados de maneira correta em uma aplicação, a beneficia com a
arquitetura e padrões da própria Web. \cite{rest-principios-e-boas-praticas}.

\end{citacao}

Conforme descrito por \cite{rest-principios-e-boas-praticas}, a aplicação
gerencia informação que pode ser tratada como recurso. Este recurso que é
gerenciado pela aplicação deve possuir uma identificação única para que seja
possível diferenciar o recurso a ser manipulado em uma requisição.

No protocolo HTTP os recursos podem ser identificados com
URI's (Uniform Resource Identifier) - também conhecido como endpoint.
Desta forma, conforme exibido no \autoref{classe-ticket-reservation-reservation-controller},
a aplicação criada possui o seguinte endpoint:

% unbreak page não quebra página
% https://tex.stackexchange.com/questions/73231/avoid-page-breaks-in-lstlistings
\begin{minipage}{\linewidth}
\begin{lstlisting}[basicstyle=\ttfamily]
http://localhost:8080/events/{idEvent}/tickets
(1)    (2)       (3) (4)     (5)      (4)
\end{lstlisting}
\end{minipage}

\index{enumerate}Este endpoint espera uma requisição seguindo o protocolo HTTP
com o método \textbf{POST}, onde cada parte é explicada abaixo. O seu objetivo é
realizar a reserva de um lugar em um determinado evento para um usuário.

\begin{enumerate}

  \item esquema ou protocolo da requisição;

  \item endereço da aplicação;

  \item porta de acesso à aplicação;

  \item nome do recurso único da aplicação (URI);

  \item identificação do evento que será feita a reserva.

\end{enumerate}

\section{Banco de dados}\label{banco-de-dados}

A aplicação precisa de um banco de dados que tenha resposta rápida para leitura e
escrita para uma quantidade limitada de registros - que é a quantidade disponível
de ingressos. Por isso, o banco de dados utilizado foi o Redis que tem o funcionamento
em memória e possui a opção de trabalhar em cluster - possibilitando a escalabilidade,
conforme \citeonline{redis-cluster-specification}.

%\subsection{ACID}

Um banco de dados em memória provê uma excelente taxa de leitura e escrita, conforme
\citeonline{a-performance-evaluation-of-in-memory-databases}.
Com isso, é possível responder com as reservas de ingressos de maneira que todas elas
sejam disponibilizadas até que se esgotem as vagas.

\subsection{Modelagem}

Há um registro no Redis que guarda o código do evento e a quantidade de ingressos
disponíveis. A partir daí é guardado o outro registro que possui a chave do evento
e o valor é uma lista com todos os códigos de ingressos disponíveis. Para reservar
um ingresso é necessário executar o comando no Redis para remover um dos códigos
de reserva disponíveis. Com isso, é subtraído um ingresso da lista do evento. Quando
não houver mais ingressos, uma resposta vazia é retornada pelo Redis.
