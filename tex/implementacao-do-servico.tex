\chapter{Implementação do serviço}

Para realizar testes de uma situação real referente às teorias citadas, foi
criado um serviço que registra reservas de ingressos. O serviço possui código
fonte open-source e pode ser acessado em
\url{https://github.com/andreformento/term-paper}.

\section{Código fonte}
Este serviço é uma aplicação escrita na linguagem Java na versão 8. O paradigma
da implementação do código fonte foi Orientação à Objetos. Porém, possui alguns
princípios de programação funcional - como é o caso da técnica de imutabilidade.

\index{alíneas}\index{subalíneas}\index{incisos}Conforme
\cite{does-immutability-really-mean-thread-safety} e \cite{java-doc-immutable-objects},
essa técnica de imutabilidade traz vários benefícios:

\begin{alineas}

  \item Objetos imutáveis facilitam a programação concorrente;

  \item Mudança indesejada em códigos dispersos;

  \item Simplicidade de manutenção do código;

  \item Como a implementação foi em Java, traz benefícios quanto ao algorítimo do
        Garbage Collector.

\end{alineas}

\lstset{language=Java,keywordstyle={\bfseries \color{blue}}}

O primeiro ponto é o mais relevante para a solução, visto que, concorrencia
eficiente é um dos pontos mais importantes. Com a palavra chave \textbf{final} é
possível definir que os atributos da classe não poderão ser alterados depois
de inicializados, conforme exemplo do \autoref{classe-ticket}.

\begin{lstlisting}[label=classe-ticket,caption=Classe Ticket em Java]
public class Ticket implements Serializable {

    @NotNull
    private final String idEvent;

    @NotNull
    private final String idUser;

    @ConstructorProperties({"idEvent", "idUser"})
    public Ticket(String idEvent, String idUser) {
        this.idEvent = idEvent;
        this.idUser = idUser;
    }
    // getters, setters, equals...
}
\end{lstlisting}

%\section{API}
%rest, endpoints
