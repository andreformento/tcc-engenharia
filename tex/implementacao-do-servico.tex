\chapter{Implementação do serviço}

Para realizar testes de uma situação real referente às teorias citadas, foi
criado um serviço que registra reservas de ingressos. O serviço possui código
fonte open-source e pode ser acessado em
\url{https://github.com/andreformento/term-paper}.

\section{Código fonte}
Este serviço é uma aplicação escrita na linguagem Java na versão 8. O paradigma
da implementação do código fonte foi Orientação à Objetos. Porém, possui alguns
princípios de programação funcional - como é o caso da técnica de imutabilidade.

\index{alíneas}\index{subalíneas}\index{incisos}Conforme
\cite{does-immutability-really-mean-thread-safety} e \cite{java-doc-immutable-objects},
essa técnica de imutabilidade traz vários benefícios:

\begin{alineas}

  \item Objetos imutáveis facilitam a programação concorrente;

  \item Mudança indesejada em códigos dispersos;

  \item Simplicidade de manutenção do código;

  \item Como a implementação foi em Java, traz benefícios quanto ao algorítimo do
        Garbage Collector.

\end{alineas}

\lstset{language=Java,keywordstyle={\bfseries \color{blue}}}

O primeiro ponto é o mais relevante para a solução, visto que, concorrencia
eficiente é um dos pontos mais importantes. Com a palavra chave \textbf{final} é
possível definir que os atributos da classe não poderão ser alterados depois
de inicializados, conforme exemplo do \autoref{classe-ticket}.

\begin{lstlisting}[label=classe-ticket,caption=Classe Ticket em Java]
public class Ticket implements Serializable {

    @NotNull
    private final String idEvent;

    @NotNull
    private final String idUser;

    @ConstructorProperties({"idEvent", "idUser"})
    public Ticket(String idEvent, String idUser) {
        this.idEvent = idEvent;
        this.idUser = idUser;
    }
    // getters, setters, equals...
}
\end{lstlisting}

\subsection{Spring Boot Framework}

Foi utilizado um framework em Java que possibilita a rápida criação de aplicações.
Conforme \cite[8]{spring-boot-reference-guide}, o Spring Boot Framework torna
produtiva a criação de serviços que rodam de maneira independente, isolada
e com várias opções de acoplar outras bibliotecas que são comuns no desenvolvimento
de software, como integração com banco de dados, sistemas de mensageria, geração
de documentação a partir do código, etc.

%\section{Banco de dados}
%O banco de dados escolhido foi o MongoDB

\subsection{API}

A aplicação desenvolvida gera uma API REST (\autoref{rest}) que é responsável por fazer uma reserva
no evento solicitado. Desta forma, este serviço atende o escopo de reserva de lugar no
evento. Conforme exibido no \autoref{classe-ticket-controller}, existe um endpoint responsável
por fazer essa reserva. Ele tem um contrato que permite que seja informado o evento e
o usuário para realizar a reserva, simplificando a sua utilização.

\index{alíneas}\index{subalíneas}\index{incisos}
Conforme descrito por \cite{mark-richards-software-architecture-patterns},
foi utilizada a arquitetura por camadas para organizar as classes.
A classe \textbf{TicketController} está na camada \textbf{presentation} e tem a
responsabilidade de receber requisições, converter para o padrão utilizado na
aplicação e chamar a classe responsável pela lógica.
Esta classe possui dependência de duas classes:

\begin{alineas}

  \item \textbf{TicketService}

  \begin{alineas}
     \item Fica na camada \textbf{business};
     \item Responsável por lidar com as regras de negócio/lógica do ingresso;
     \item É ela a reponsável por lidar com as validações;
  \end{alineas}

  \item \textbf{TicketMapper}

  \begin{alineas}
     \item Fica na camada \textbf{presentation};
     \item Responsável por converter o que vem de fora da aplicação e transformar
           no padrão adotado dentro da aplicação;
     \item E também o inverso do item anterior: o que for gerado dentro da aplicação -
           seguindo um padrão definido, ela é reponsável por converter
           os dados a serem expostos de modo que mantenha um contrato pré-determinado
  \end{alineas}

\end{alineas}

\begin{lstlisting}[label=classe-ticket-controller,caption=Classe TicketController em Java]
@RestController
@RequestMapping("/events")
public class TicketController {

    private final TicketService service;
    private final TicketMapper mapper;

    public TicketController(TicketService service, TicketMapper mapper) {
        this.service = service;
        this.mapper = mapper;
    }

    @PostMapping
    @RequestMapping("/{idEvent}/tickets")
    public HttpEntity<Resource<TicketResponse>> booking(
        @PathVariable("idEvent") String idEvent,
        @RequestBody final TicketRequest ticketRequest
    ) {

        return mapper.mapToResponse(service.booking(mapper.mapFromRequest(idEvent, ticketRequest)));
    }

}
\end{lstlisting}

\subsection{REST}\label{rest}

\begin{citacao}

REST é um dos modelos de arquitetura que foi descrito por
Roy Fielding \cite[5]{rest-roy-thomas-fielding},
um dos principais criadores do protocolo HTTP, em sua tese de doutorado
e que foi adotado como o modelo a ser utilizado na evolução da arquitetura
do protocolo HTTP.

[...] REST na verdade pode ser considerado como um conjunto de princípios,
que quando aplicados de maneira correta em uma aplicação, a beneficia com a
arquitetura e padrões da própria Web. \cite{rest-principios-e-boas-praticas}.

\end{citacao}

Conforme descrito por \cite{rest-principios-e-boas-praticas}, as aplicações
gerenciam informação que pode ser tratada como recurso. Este recurso que é
gerenciado pela aplicação deve possuir uma identificação única para que seja
possível diferenciar o recurso a ser manipulado em uma requisição.

No protocolo HTTP os recursos podem ser identificados com
URI's (Uniform Resource Identifier) - também conhecido como endpoint.
Desta forma, conforme exibido no \autoref{classe-ticket-controller},
a aplicação criada possui o seguinte endpoint:

\begin{lstlisting}[basicstyle=\ttfamily]
http://localhost:8080/events/\{idEvent\}/tickets
(1)    (2)       (3) (4)     (4)
\end{lstlisting}

\index{enumerate}Este endpoint espera uma requisição seguindo o protocolo HTTP
com o método \textbf{POST}, onde cada parte é explicada abaixo:

\begin{enumerate}

  \item Objetos imutáveis facilitam a programação concorrente;

  \item Mudança indesejada em códigos dispersos;

  \item Simplicidade de manutenção do código;

  \item Como a implementação foi em Java, traz benefícios quanto ao algorítimo do
        Garbage Collector.

\end{enumerate}

O seu objetivo é realizar a reserva de um lugar em um determinado evento para um
usuário.
