\chapter{Balanceador de carga}

Balanceamento de carga é uma técnica que tem por objetivo a distribuição de carga de trabalho uniformemente entre dois ou
mais computadores. Essa técnica também pode ser utilizada em enlaces de redes e discos rígidos. O mecanismo visa atingir
escalabilidade do sistema e propiciar alta disponibilidade de acesso e respostas de requisições. Ou seja, a medida que a
demanda pelos serviços, recursos ou o número de usuários conectados aumentam, mais máquinas são disponibilizadas para incorporar
o conjunto e auxiliar no processamento dos dados.


O balanceamento de rede, consiste na técnica de distribuição de tráfego de rede entre servidores. Essa solução visa entregar
caminhos alternativos a fim de descongestionar os acessos atendendo um número maior de requisições. A segunda geração dos
balanceadores de carga são seguidos pelos sistemas proprietários baseados em aplicativos. Se comportam de forma transparente
as aplicações, ou seja, quando um único servidor não comporta toda carga gerada, é possível iniciar uma nova instância em
outro servidor e balancear a aplicação entre eles.


Utilizando-se do balanceamento de carga, o trabalho propõem tratar um problema que está relacionado com a exponencial
quantidade de requisições recebidas pelos servidores que hospedam sites de vendas de ingresso. A sobrecarga de um
sistema pode ocorrer de maneira linear - com o passar dos anos, por exemplo; ou ainda, exponencial - algum pico de uso
que pode ocorrer com o rápido aumento de requisições em questão de minutos.
No caso do problema da reserva de ingressos, esses picos podem ocorrer regularmente a cada abertura da venda de ingressos.
Com estes pico, cada vez que houver aumento de processamento das requisições e os servidores passarem a entrarem em
estado de sobrecarga, mais servidores podem ser disponibilizados para atenderem a alta demanda e o balanceador de carga
ficaria responsável por distribuir a carga para esses novos servidores também, tornando assim a estrutura escalável
horizontalmente.

Sugere-se como uma solução que atende as exigências citadas a utilização de balanceamento de carga baseado em hardware.
Essa tecnologia tem por base dispositivos de rede. Esses diapositivos são neutros em relação aos aplicativos, dessa forma
consegue-se atuar em qualquer plataforma. Em resumo, os componentes oferecem um "servidor virtual" para todos os serviços
que requerem acesso, quando o servidor aceita a requisição, encaminha a mesma para o "servidor real" disponível para atender
e processar os dados. Suas vantagens frente aos demais, é o controle dos servidores, sua disponibilidade, atividade e
operação, monitorado do estado do hardware, isso garante que o sistema irá responder todas as requisições sem perder
desempenho. Com esse gerenciamento também é possível detectar quando os servidores recebem baixo número de acessos, nesse
estado algumas máquinas são desativadas temporariamente, visando economia de energia.
