\chapter{Balanceador de carga}


A necessidade constante de mais disponibilidade em serviços, impulsionou a criação de uma técnica computacional 
que consegue atender uma exigência cada vez mais comum e em muitos casos indispensável. Quando o balanceamento 
de carga foi implementado, esperava-se sanar um problema que causava uma série de inconsistências. A técnica pode
ser usada em vários componentes. Ao exemplo do balanceamento de armazenamento (storage), que possibilita o 
armazenamento, e distribuição de acesso compartilhados em vários discos. Sua eficiência e a redução no custo das 
peças possibilitaram muitos aderirem ao sistema de distribuição carga através de discos  que hoje podem ser 
encontrados de computadores domésticos até supercomputadores. Quando fala-se em balanceamento de rede, consiste 
na técnica de distribuição de tráfego de rede entre servidores.


Encontrado em todas as camadas do modelo OSI, essa solução visa entregar caminhos alternativos a fim de descongestionar
os acessos atendendo um número maior de requisições. O balanceamento de CPU é construido com o intuito de equilibrar o
processamento de dados através do compartimento computacional. Usado em sistemas de processamento distribuído e 
organizado de tal forma que atende os mais altos índices de solicitações de processamento de informações. Os 
balanceadores que fora apresentados são normalmente encontrados em grandes centros de processamento. Organizações 
que atuam provendo conteúdo na rede mundial de computadores, centros de pesquisas, instituições de ensino e no âmbito
governamental.


A alta escalabilidade está intrinsecamente ligada ao balanceamento de carga que está implementado, isso porque a medida 
que a aplicação alocada expande-se ao ponto de um servidor não conseguir atender a demanda de processamento adequadamente,
este necessita escalar processamento para outros servidores. O balanceamento é um recurso utilizado para que a alta
disponibilidade tenha eficácia e mantenha serviços ativos. Uma vez escalado o processamento, este pode apresentar lentidão
e baixo desempenho, um dos problemas que pode-se identificar com esse comportamento é a infraestrutura instalada. 
O alto desempenho é obtido se o balanceamento de carga dividir as tarefas dos serviços de forma inteligentes, ou seja, 
que consiga direcionar as novas requisições aos servidores mais ociosos, com um gerenciamento deficiente cria-se gargalos
na aplicação e ônus aos equipamentos.


Juntamente aos já mencionados, a alta disponibilidade e a recuperação de quedas, também são benefícios adquiridos quando
fala-se de balanceamento de carga. Devidamente configurado, o balanceamento é capaz de manter disponível os serviços, 
mesmo que o sistema tenha sofrido falhas ao longo da execução. Os servidores que apresentam falhas são isolados, 
busca-se outros servidores desocupados que possam assumir as requisições. Contornado o problema de disponibilidade, 
o sistema emite alertas para que o servidor defeituoso possa ser reparado.


Ao que compete o trabalho proposto, uma das áreas que mais utilizam o balanceamento de carga estão nas empresas que 
possuem um elevado tráfego de informações via web. Estas operam em um nível de criticidade muito alto, o downtime em 
que estão dispostas é o menor possível.


A escolha de webfarm é aplicável e uma das mais indicadas para contornar um cenário de sobrecarga. Adicionando aos 
proventos que relata-se  nesse trabalho, o balanceamento de carga representa um modo de implementação flexível. 
Pode-se implantar em hardware, software ou ambos. Em hardware, pode-se encontrar técnicas que agregam múltiplos 
links para que atuem em um único link lógico.


Com conxeões SSL, é possível no final do processo atuar em transições de tráfego entre formulários criptografados e 
não criptografados.Tem-se como opção de balanceamento de carga que utiliza
o hardware, um mecanismo de traduzir endereços NAT - Network Address Translation . Esse mecanismo recebe um link de 
endereço notável e redistribui para um conjunto de computadores que possuem endereço denominados locais.


Avançando em recursos que lidam com o hardware, existe ocasiões que viabiliza o uso de roteamento direto. Esse artifício
permite a interação direta de servidor e usuário solicitante, excluindo a necessidade dos pacotes de saída passarem por
roteadores, cabendo apenas aos roteadores o processamento de pacotes de entrada.


Quando necessita-se conhecer sobre balanceamento de carga por software, servidores DNS - domain name server, possibilitam
a instalação de um equilibrador de carga. Entende-se por carga de trabalho do DNS, os pedidos de resoluções de nomes que 
são distribuídos para um determinado destino, esses pedidos são compartilhados por dois ou mais servidores de nomes.


Discutindo-se sobre a ferramentas apresentadas, tem se ainda a escolha de adotar modelos híbridos que lidam com ambas
soluções. Contudo, a gama de instrumentos que são ofertados,  demonstra o grau de aplicabilidade do balanceamento de 
carga, respondendo com eficiência as exigências dos seus utilizadores. 

