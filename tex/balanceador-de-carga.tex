\chapter{Balanceador de carga}

Balanceamento de carga é uma técnica que tem por objetivo a distribuição de carga de trabalho uniformemente entre dois ou
mais computadores. Essa técnica também pode ser utilizada em enlaces de redes e discos rígidos. O mecanismo visa atingir
escalabilidade do sistema e propiciar alta disponibilidade de acesso e respostas de requisições, ou seja, a medida que a
demanda pelos serviços, recursos ou o número de usuários conectados aumentam, mais máquinas são acionadas para incorporar
o conjunto e auxiliar no processamento dos dados.


O balanceamento de rede, consiste na técnica de distribuição de tráfego de rede entre servidores. Essa solução visa entregar
caminhos alternativos a fim de descongestionar os acessos atendendo um número maior de requisições. A segunda geração dos
balanceadores de carga são seguidos pelos sistemas proprietários baseados em aplicativos. Se comportam de forma transparente
as aplicações, ou seja, quando um único servidor não comporta toda carga gerada, é possível iniciar uma nova instância em
outro servidor e balancear a aplicação entre eles.


Utilizando-se do balanceamento de carga o trabalho propõem tratar um problema que está relacionado com a exponencial
quantidade de requisições recebidas pelos servidores que hospedam sites de vendas de ingresso. O agravante está no
desconhecimento dos utilizadores de TI sobre quantidade de erros que acorre caso disponibilizem algum serviço para o
público, tendo uma estrutura insuficiente para atender altos picos de acessos.

O problema poderia demandar uma série de outros problemas para o provedor, isso porque este perderia uma quantidade muita
alta de clientes, muitos desistiriam ou iriam adquirir o produto de outra forma. A confiabilidade da empresa web seria
abalada negativamente, pois diante da frustração, os usuários poderiam passar uma imagem inadequada e que implicaria a perda
de possíveis novos clientes. Consumidores de produtos que compram pela internet, possuem um comportamento que difere em
alguns casos dos compradores de lojas físicas. Eles estão acostumados com um atendimento simples, rápido e sem burocracia.
Quando desejam adquirir algo, seus possíveis vendedores devem oferecer uma experiência que abrange tudo que ele irá
necessitar para comprar o produto, alguma inatividade no site pode significar a perda de uma venda.


Ao que compete a empresa contratante, o provedor do serviço, também sofreria e sua imagem como empresa confiável seria
duramente criticada. A razão pelo efeito dominó está que a única forma de colocar os produtos acessíveis ao público é de
forma virtual. Inconsistências no acesso dos clientes podem ser comparados as portas fechadas de uma loja física.


Sugere-se como uma solução que atende as exigências citadas a utilização de balanceamento de carga baseado em hardware.
Essa tecnologia tem por base dispositivos de rede. Esses diapositivos são neutros em relação aos aplicativos, dessa forma
consegue-se atuar em qualquer plataforma. Em resumo, os componentes oferecem um “servidor virtual” para todos os serviços
que requerem acesso, quando o servidor aceita a requisição, encaminha a mesma para o “servidor real” disponível para atender
e processar os dados. Suas vantagens frente aos demais, é o controle dos servidores, sua disponibilidade, atividade e
operação, monitorado do estado do hardware, isso garante que o sistema irá responder todas as requisições sem perder
desempenho. Com esse gerenciamento também é possível detectar quando os servidores recebem baixo número de acessos, nesse
estado algumas máquinas são desativadas temporariamente, visando economia de energia e preservação dos equipamentos.

