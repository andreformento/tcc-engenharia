\chapter{Balanceador de carga}\label{balanceador-de-carga}

Pensando no problema da quantidade de carga que um servidor aguenta, ou seja, quantas requisições ele consegue atender,
surge a solução de ter mais servidores disponíveis para atender essas requisições.
Para que haja distribuição dessas requisições é necessário um mecanismo, e é isso que o balanceador de carga se
propõe a resolver.

\begin{citacao}
Balanceamento de carga consiste em uma técnica para distribuição de carga de trabalho uniformemente entre dois ou mais
computadores, enlaces de rede, UCPs, discos rígidos ou outros recursos. Intuito pelo qual visa o aperfeiçoamento da
utilização dos recursos, a maximização do desempenho, a diminuição do tempo de resposta e ainda evitar sobrecarga.
Devido a redundância de múltiplos componentes a confiabilidade
aumenta \cite{o-que-e-e-para-que-serve-o-balanceamento-de-carga}.
\end{citacao}


\section{Balanceador de carga no serviço de reserva de ingressos}

Utilizando-se do balanceamento de carga, o trabalho propõem tratar um problema que está relacionado com a exponencial
quantidade de requisições recebidas pelos servidores que hospedam sites de vendas de ingresso. A sobrecarga de um
sistema pode ocorrer de maneira linear - com o passar dos anos, por exemplo; ou ainda, exponencial - algum pico de uso
que pode ocorrer com o rápido aumento de requisições em questão de minutos.
No caso do problema da reserva de ingressos, esses picos podem ocorrer regularmente a cada abertura da venda de ingressos.
Com estes picos, cada vez que houver aumento de processamento das requisições e os servidores entrarem em
estado de sobrecarga, mais servidores podem ser disponibilizados para atenderem a alta demanda.
O balanceador de carga ficaria responsável por distribuir a carga para esses novos servidores,
tornando assim a estrutura escalável horizontalmente.

Sugere-se como uma camada no nível de hardware que atende as exigências citadas.
Essa tecnologia tem por base dispositivos de rede. Esses dispositivos são neutros em relação aos aplicativos, dessa forma
consegue-se atuar em qualquer plataforma. Em resumo, os componentes oferecem um "servidor virtual" para todos os serviços
que requerem acesso, quando o servidor aceita a requisição, encaminha a mesma para o "servidor real" disponível para atender
e processar os dados.
Isso garante que o sistema irá responder todas as requisições sem perder desempenho.
Com esse gerenciamento também é possível detectar quando os servidores recebem baixo número de acessos, nesse
estado algumas máquinas são desativadas temporariamente, visando economia de energia.

O serviço de reserva de ingressos poderá rodar em vários servidores para atender muitas requisições.
O balanceador de cargas ficará responsável por distribuir as requisições entre todos os servidores disponíveis.
Com o balanceador de cargas abstraindo essa estrutura, o cliente que utiliza essa infraestrutura
não notará diferença na utilização.
