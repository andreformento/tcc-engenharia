\chapter*[Conclusão]{Conclusão}
\addcontentsline{toc}{chapter}{Conclusão}

% escalabilidade
% performance
% disponibilidade
% microsserviço
% resultado
% balanceamento de cargas
% nuvem
A organização da arquitetura do trabalho através de microsserviços possibilitou
que vários aspectos fossem tratados de maneira especifica para implementar o serviço,
ao contrário do que acontece em um sistema monolítico.
A escolha do banco de dados em memória (Redis), por exemplo,
viabilizou o acesso rápido aos dados, mas não obriga que outras partes do sistema
utilizem esse mesmo tipo de banco de dados.
Conclui-se que há um desacoplamento no nível de software, pois a dependência foi
para a camada de comunicação.

O serviço implementado permite escalabilidade horizontal com
Docker e Kubernetes, ferramentas que viabilizam a eslasticidade da infraestrutura.
Com elas é possível fazer otimizações, realizando
redução ou aumento de máquinas, dependendo da demanda.
%Isso influência na performance e disponibilidade do sistema.
Com estas ferramentas há uma abstração do ambiente e é posssível colocar a aplicação
em um servidor de maneira isolada e desacoplada, tendo como único requisito
o Sistema Operacional Linux.
Ou seja, pode ser implantada em qualquer plataforma de nuvem (IaaS) que atenda
este requisito.
Conclui-se que infraestrutura ficou com baixo acoplamento no nível de hardware
e ampla possibilidade de escalabilidade para garantir performance e disponibilidade.

Os resultados apresentados mostram que houve degradação dos serviços com o aumento do
número de requisições para a aplicação rodando em um servidor.
O tempo de resposta aumentou conforme aumentou o número de requisições.
Mesmo com o aumento do tempo de resposta, a taxa de erros se manteve em 0\% em
quase todos os cenários testados.
Foi somente a partir das 30.000 requisições feitas num tempo de 33 segundos que
a aplicação começou a ter problemas e deixar responder 6\% delas.
Conclui-se que a aplicação consegue responder em 34 segundos
28.000 requisições de ingressos sem erro.
Como cada requisição feita no teste solicita 2 ingressos, foi posssível
reservar 56.000 ingressos em 34 segundos sem nenhum erro.

%Em um sistema monolítico, onde tudo está na mesma base de código,
%é necessário que todas as implementações utilizem a mesma linguagem de programação.



% microsserviços
%As vantagens que os microseviços trazem foram o motivo da escolha por esta abordagem.
%Isso possibilitou separar e implementar apenas uma parte
%de uma solução mais geral, que é a venda de ingressos.
%Foi possível também que os testes de performance fossem realizados num ponto específico
%e deixado claro o comportamento no momento da vendas de ingressos.
%Esses dados geram estimativas inclusive para tomadas de decisão quanto aos custos
%que uma venda de ingressos pode gerar - e a escolha de qual plataforma utilizar.

% resultados
%Os resultados apresentados mostram que houve degradação dos serviços com o aumento do
%número de requisições para a aplicação rodando em um servidor.
% escalabilidade / balanceamento de cargas
%Usando a abordagem de escalabilidade horizontal e o balanceamento de cargas,
%é possível que mais servidores sejam disponibilizados para atender demandas altas.

%Como o serviço de reserva de ingressos está desacoplado em um serviço de contexto bem
%definido, é possível lidar com ele de maneira independente.
% nuvem
%Isso em um ambiente de nuvem (IaaS) torna possível a auto escalabilidade com baixo esforço.
%Com o uso de ferramentas que auxiliam na implantação, como Docker e Kubernetes,
%é possível automatizar o ambiente para que se comporte de maneira otimizada -
%quando há grande demanda, vários servidores são disponibilizados, mas no momento
%de baixa utilização, várias máquinas podem ser desligadas.
% performance / disponibilidade
%Isso influência no desempenho do sistema.
%A performance e a disponibilidade é garantida e a degradação é minimizada.

% conclusão final :D
%Para o usuário final a experiência de comprar um ingresso em um sistema desses
%é positiva, visto que, mesmo que vários outros usuários estão tentando
%fazer aquela mesma compra, ele não é afetado e consegue concluir de maneira
%satisfatória o processo.
