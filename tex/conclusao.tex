\chapter*[Conclusão]{Conclusão}
\addcontentsline{toc}{chapter}{Conclusão}

% microserviços
As vantagens que os microseviços trazem foram o motivo da escolha por esta abordagem.
Isso possibilitou separar e implementar apenas uma parte
de uma solução mais geral, que é a venda de ingressos.
Foi possível também que os testes de performance fossem realizados num ponto específico
e deixado claro o comportamento no momento da vendas de ingressos.
Esses dados geram estimativas inclusive para tomadas de decisão quanto aos custos
que uma venda de ingressos pode gerar - e a escolha de qual plataforma utilizar.

% resultados
O resultados apresentados mostram que houve degradação dos serviços com o aumento do
número de requisições.
% escalabilidade / balanceamento de cargas
Porém, usando a abordagem de escalabilidade horizontal e o balanceamento de cargas,
é possível que mais servidores sejam disponibilizados para atender essa alta demanda.

Como o serviço de reserva de ingressos está desacoplado em um serviço de contexto bem
definido, é possível lidar com ele de maneira independente.
% nuvem
Isso em um ambiente de nuvem (IaaS) torna possível a auto escalabilidade com baixo esforço.
Com o uso de ferramentas que auxiliam na implantação, como Docker e Kubernetes,
é possível automatizar o ambiente para que se comporte de maneira otimizada -
quando há grande demanda, vários servidores são disponibilizados, mas no momento
de baixa utilização, várias máquinas podem ser desligadas.
% performance / disponibilidade
Isso influência no desempenho do sistema.
A performance e a disponibilidade é garantida e a degradação é minimizada.

% conclusão final :D
Para o usuário final a experiência de comprar um ingresso em um sistema desses
é positiva, visto que, mesmo que vários outros usuários estão tentando
fazer aquela mesma compra, ele não é afetado e consegue concluir de maneira
satisfatória o processo.
