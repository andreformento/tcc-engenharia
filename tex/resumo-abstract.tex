% resumo em português
\setlength{\absparsep}{18pt} % ajusta o espaçamento dos parágrafos do resumo
\begin{resumo}

  % https://blog.mettzer.com/resumo-abstract-tcc/

  Quando um sistema de ingressos abre as vendas para o público é comum
  que haja muita demanda, ou seja, um pico de requisições.
  Este trabalho apresenta uma abordagem com microsserviços para tratar
  especificamente do processo de reserva de ingressos.
  O objetivo é tornar a arquitetura elástica para que nesses momentos
  de grande demanda, mais recursos fiquem disponíveis;
  e, quando a demanda for reduzida, haja otimização -
  garantindo assim escalabilidade, performance e disponibilidade.
  A conclusão mostra que a abordagem de microsserviços garantiu a
  abstração da camada de software, enquanto que a utilização de
  ferramentas adequadas permitiu a abstração da camada de hardware,
  e também, que toda a infraestrutura seja implantada em nuvem.
  Porém, as métricas mostraram que o serviço degradou-se ao receber muitas requisições.

  \textbf{Palavras-chaves}: Escalabilidade, Performance, Disponibilidade, Microsserviço

% escalabilidade
% performance
% disponibilidade
% microsserviço
% resultado
% balanceamento de cargas
% nuvem

\end{resumo}

% resumo em inglês
\begin{resumo}[Abstract]
 \begin{otherlanguage*}{english}
   This is the english abstract.

   \vspace{\onelineskip}

   \noindent
   \textbf{Key-words}: Scalability, Performance, Availability, Microservice
 \end{otherlanguage*}
\end{resumo}
