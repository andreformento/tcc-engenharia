%\chapter{Trabalhos futuros}\label{trabalhos-futuros}

\chapter*[Trabalhos futuros]{Trabalhos futuros}\label{trabalhos-futuros}
\addcontentsline{toc}{chapter}{Trabalhos futuros}

% escrever o serviço em outra linguagem

Em um contexto de um sistema de venda de ingressos, poderiam ser feitos serviços que atendessem
outras necessidades de negócio, como serviço de venda do ingresso, serviço de listagem de eventos
e assim por diante.
Também poderiam ser abordados temas como segurança, controles de acesso e serviço de usuário.

Poderia ser feito um trabalho com uma abordagem diferente para lidar com a venda de ingressos.
Há outras linguagens que lidam com concorrência de maneira mais otimizada,
proporcionando assim uma melhora no tempo de resposta da aplicação com a mesma quantidade
de recursos.
O Whatsapp, por exemplo, utiliza a linguagem Erlang \citeonline{why-whatsapp-used-erlang} que
atende milhares de requisições por segundo.

Há também uma abordagem de microsserviços utilizando o modelo de coreografia, onde seria possível
lidar com o atendimento de muitas requisições de maneira assíncrona, conforme dito por
\cite{scaling-microservices-event-stream}.

Atualmente não é possível que um usuário final faça uma iteração com a aplicação. O desenvolvimento
de uma aplicação frontend, como móbile ou website, poderia explorar o aspecto de usabilidade
do sistema.
