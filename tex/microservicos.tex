\chapter{Microsserviços}\label{microservicos}

Como expressado por
\cite{building-microservices}, ``Microsserviços são pequenos serviços
autônomos que trabalham em conjunto``.

\begin{citacao}
Em suma, o estilo arquitetônico do microsserviço é uma abordagem para desenvolver
uma única aplicação como um conjunto de pequenos serviços, cada um executando em seu próprio processo e se comunicando
com mecanismos leves, muitas vezes uma API de recursos HTTP. Esses serviços são construídos em torno de recursos de
negócios e implementáveis independentemente por meio de maquinaria de implantação totalmente automatizada.
Existe um mínimo de gerenciamento centralizado desses serviços, que pode ser escrito em diferentes linguagens de
programação e usar diferentes tecnologias de armazenamento de dados \cite{martin-fowler-microservices}.
\end{citacao}

Com microsserviços é possível escalar apenas algumas partes do sistema, diminuindo
o custo e esforço para aumentar o poder de processamento para atender muitas requisições.

Pode haver um serviço que cuida do contexto do usuário; outro serviço que lida com a
lista de eventos disponíveis; outro serviço responsável por realizar reserva de vaga de
cada evento; e assim por diante. Ou seja, cada um desses serviços pode ter a sua a quantidade
de requisições diferentes umas das outras, mas os serviços poderão ser escalados de forma
independente e conforme a necessidade.

Um exemplo real pode ser objservado em um e-commerce.

\begin{citacao}

Ao criarmos a Editora Casa do Código, decidimos seguir outro caminho para o e-commerce.
Lá não podemos correr o risco de ficar com a loja fora do ar caso alguma funcionalidade
periférica falhe, então quebramos nossa arquitetura em serviços menores. Dessa forma, há
um sistema principal, que é a loja e está hospedada no Shopify, e vários outros sistemas
que gravitam em torno dela. Temos uma aplicação que faz a liberação dos e-books para os
clientes, outro que contabiliza os royalties para autores, outro para cuidar da logística
de envio dos livros impressos para o cliente, um painel de visualização dos livros comprados,
um para fazer a liberação de vale presentes e outro para promoções.

Dessa forma, quando um evento acontece na nossa loja online, os diferentes sistemas precisam
ser notificados. Para essas notificações usamos requisições HTTP, assim, quando uma compra
é confirmada na loja online, todos os sistemas recebem em um Endpoint essa requisição HTTP,
contendo um JSON com todos os dados da compra. Cada sistema decide o que fazer com as informações
recebidas, de acordo com a necessidade. Por exemplo, o sistema de liberação dos e-books verifica
se a compra possuía e-books e gera os links de download para o comprador, enquanto que o sistema
de logística já dá baixa no estoque quando a compra é de um impresso, além de notificar as pessoas
quando algum livro está ficando com estoque crítico. Essa característica importante para o maior
desacoplamento dos serviços é conhecida como Smart endpoints, dumb pipes, que em uma tradução
livre pode ser entendida como Endpoints inteligentes e fluxos
simples \cite{arquitetura-de-microservicos-ou-monolitica}.

\end{citacao}


Considerando que a reserva de ingressos é um contexto bem definido, pode haver um serviço que cuide
especificamente desta parte. Desta forma, é possível isolar e pensar de forma objetiva em como
lidar com esse problema.

\section{API Gateway}\label{api-gateway}

A quantidade de serviços que um sistema tem pode ser grande.
É importante abstrair essa complexidade para quem usa o sistema.
Uma forma de fazer isso é utilizando o conceito de API Gateway.
Com isso, há um ponto único de acesso pelos clientes, sejam eles
mobile, webiste, ou outro sistema.
Os serviços passam a ser acessados de maneira indireta.
Também é responsabilidade do API Gateway lidar com a segurança
envolvendo os serviços.

Desta forma, o serviço de reserva de ingressos terá uma camada a mais
para ser acessada, abstraindo a segurança para esta camada.

\section{Comparação entre microsserviços e sistemas monolíticos}\label{microservicos-monoliticos}

Um sistema é considerado monolítico quando todas as implementações estão em um mesma base de código
e há forte dependência entre os módulos no nível de código.


\begin{citacao}
A arquitetura monolítica é um padrão comumente usado para o desenvolvimento de aplicações corporativas.
Esse padrão funciona razoavelmente bem para pequenas aplicações, pois o desenvolvimento, testes e
implantação de pequenas aplicações monolíticas é relativamente simples \cite{arquitetura-monolitica}.
\end{citacao}


Quando há um único sistema, costuma haver separação de responsabilidades através de bibliotecas.
Essas bibliotecas passam a ser usadas em várias partes do sistema. Porém, elas estão fortemente
acopladas à solução.
Mesmo contendo várias bibliotecas, todas estarão atreladas ao mesmo processo.
Uma vez que se faz alterações em qualquer delas, resulta em uma implantação do sistema inteiro.

Nos microsserviços a independência de sua implantação é algo tido como uma das razões para
utilizá-los ao invés das bibliotecas.

Com o aumento do desenvolvimento de aplicações cresce a base de código.
Essas bases aumentam tanto de tamanho como de complexidade,
tornando-se difícil fazer manutenções ou criar melhorias.

A implementação utilizando o conceito de microseviços trata de problemas na arquitetura do sistema.
A abstração passa da camada de código para a camada de comunicação. Desta forma, ao invés de haver
integração através do código - uma função chamando outra, a integração ocorre através de chamadas de
APIs.

Com um contrato bem definido, é possível que um serviço faça utilização do outro.
Com isso, não importa em qual linguagem é desenvolvido o serviço.
Mantendo o contrato, podem haver alterações, manutenções, melhorias, reimplementações,
e assim por diante, sem afetar quem utiliza aquele serviço.

As vantagens em microsserviços abrangem inclusive a equipe atuante no desenvolvimento, já que com nesse paradigma,
elas planejam uma atuação em serviços diferentes, sem que essas mudanças afetem os demais integrantes.
Adere a escalabilidade de forma seletiva, possibilitando escalar somente a parte do sistema que necessita de ajustes.
A divisão do sistema também beneficia a velocidade de desenvolvimento, o código é de fácil entendimento e
compilam mais rápido por serem soluções relativamente pequenas, conforme mencionado em um artigo escrito pelos
arquitetos da Amazon AWS, \citeonline{deploying-java-microservices-on-amazon-ec2-container-service}.

\begin{citacao}
No geral, os microservices em contentores são mais rápidos de desenvolver,
mais fáceis de iterar e mais económicos para manter e
proteger \cite{deploying-java-microservices-on-amazon-ec2-container-service}.
\end{citacao}
