\chapter{@@Microserviços}\label{microservicos}

Como expressado por
\cite{building-microservices}, ``Microserviços são pequenos serviços
autônomos que trabalham em conjunto``.

\begin{citacao}
Em suma, o estilo arquitetônico do microserviço é uma abordagem para desenvolver
uma única aplicação como um conjunto de pequenos serviços, cada um executando em seu próprio processo e se comunicando
com mecanismos leves, muitas vezes uma API de recursos HTTP. Esses serviços são construídos em torno de recursos de
negócios e implementáveis independentemente por meio de maquinaria de implantação totalmente automatizada.
Existe um mínimo de gerenciamento centralizado desses serviços, que pode ser escrito em diferentes linguagens de
programação e usar diferentes tecnologias de armazenamento de dados \cite{martin-fowler-microservices}.
\end{citacao}

Com microserviços é possível escalar apenas algumas partes do sistema, diminuindo
o custo e esforço para aumentar o poder de processamento para atender muitas requisições.

Pode haver um serviço que cuida do contexto do usuário; outro serviço que lida com a
lista de eventos disponíveis; outro serviço responsável por realizar reserva de vaga de
cada evento; e assim por diante. Ou seja, cada um desses serviços pode ter a sua a quantidade
de requisições diferentes umas das outras, mas os serviços poderão ser escalados de forma
independente e conforme a necessidade.

Um exemplo real pode ser objservado em um e-commerce.

\begin{citacao}

Ao criarmos a Editora Casa do Código, decidimos seguir outro caminho para o e-commerce.
Lá não podemos correr o risco de ficar com a loja fora do ar caso alguma funcionalidade
periférica falhe, então quebramos nossa arquitetura em serviços menores. Dessa forma, há
um sistema principal, que é a loja e está hospedada no Shopify, e vários outros sistemas
que gravitam em torno dela. Temos uma aplicação que faz a liberação dos e-books para os
clientes, outro que contabiliza os royalties para autores, outro para cuidar da logística
de envio dos livros impressos para o cliente, um painel de visualização dos livros comprados,
um para fazer a liberação de vale presentes e outro para promoções.

Dessa forma, quando um evento acontece na nossa loja online, os diferentes sistemas precisam
ser notificados. Para essas notificações usamos requisições HTTP, assim, quando uma compra
é confirmada na loja online, todos os sistemas recebem em um Endpoint essa requisição HTTP,
contendo um JSON com todos os dados da compra. Cada sistema decide o que fazer com as informações
recebidas, de acordo com a necessidade. Por exemplo, o sistema de liberação dos e-books verifica
se a compra possuía e-books e gera os links de download para o comprador, enquanto que o sistema
de logística já dá baixa no estoque quando a compra é de um impresso, além de notificar as pessoas
quando algum livro está ficando com estoque crítico. Essa característica importante para o maior
desacoplamento dos serviços é conhecida como Smart endpoints, dumb pipes, que em uma tradução
livre pode ser entendida como Endpoints inteligentes e fluxos
simples \cite{arquitetura-de-microservicos-ou-monolitica}.

\end{citacao}


Considerando que a reserva de ingressos é um contexto bem definido, pode haver um serviço que cuide
especificamente desta parte. Desta forma, é possível isolar e pensar de forma objetiva em como
lidar com esse problema.

\section{API Gateway}\label{api-gateway}

A quantidade de serviços que um sistema tem pode ser grande.
É importante abstrair essa complexidade para quem usa o sistema.
Uma forma de fazer isso é utilizando o conceito de API Gateway.
Com isso, há um ponto único de acesso pelos clientes, sejam eles
mobile, webiste, ou outro sistema.
Os serviços passam a ser acessados de maneira indireta.
Também é responsabilidade do API Gateway lidar com a segurança
envolvendo os serviços.

Desta forma, o serviço de reserva de ingressos terá uma camada a mais
para ser acessada, abstraindo a segurança para esta outra camada.


\section{Comparação entre microserviços e sistemas monolíticos}\label{microservicos-monoliticos}
A implementação de microseviços deve-se ao fato de resolver problemas na estrutura do sistema. Com o aumento exponencial
na quantidade de aplicações que são desenvolvidas, tem-se milhares de bases de códigos. Essas bases
aumentam tanto de tamanho como de complexidade, características dificultam a equipe de desenvolvimento, ao passo que a
medida que o código cresce, torna-se difícil reparar ou realizar mudanças. Esse cenário torna-se mais desafiador quando
fala-se de um sistema monolítico. Para que se entenda as vantagens de usar os microserviços, pode-se comparar a arquitetura
monolítica.


Um programa monolítico é estruturado usando desvios condicionais ou incondicionais, não fazendo uso explicito de mecanismos
auxiliares de programação que permitam uma melhor estruturação do
controle \cite{teoria-da-computacao-3ed-ufrgs-maquinas-universais-e-computabilidade}.


Nos microserviços a independência de sua implantação é algo tido como uma das razões para utilizá-los ao invés das
bibliotecas. Estas são atreladas a um único processo, mesmo contendo várias bibliotecas, todas estarão atreladas ao mesmo
processo, uma vez que se faz alterações em qualquer delas, resulta em uma implantação do aplicativo inteiro.
As vantagens em microserviços abrangem desde a equipe atuante no desenvolvimento, já que com nesse paradigma, elas planejam
uma atuação em serviços diferentes, sem que essas mudanças afetem os demais integrantes. Adere a escalabilidade de forma
seletiva, possibilitando escalar somente a parte do sistema que necessita de ajustes.
A divisão do sistema também beneficia a velocidade de desenvolvimento, o código é de fácil entendimento e
compilam mais rápido por serem soluções relativamente pequenas, conforme mencionado em um artigo escrito pelos
arquitetos da Amazon AWS, \citeonline{deploying-java-microservices-on-amazon-ec2-container-service}.

\begin{citacao}
A migração de um aplicativo monolítico para um conjunto de microserviços em contêiner pode parecer uma tarefa assustadora.
Seguindo as etapas assertivamente, você pode começar a contornar aplicativos monolíticos, aproveitando o ambiente de tempo
de execução do contêiner e iniciando o processo de reestruturação em microservices. No geral, os microservices em
contentores são mais rápidos de desenvolver, mais fáceis de iterar e mais económicos para manter e
proteger \cite{deploying-java-microservices-on-amazon-ec2-container-service}.
\end{citacao}
