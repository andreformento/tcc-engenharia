% está é uma subsection de infraestrutura em nuvem do arquivo infraestrutura.tex

\subsection{Plataformas em nuvem}

Há diversas opções de plataformas em nuvem no mercado: AWS, Google Cloud Plataform,
Openstack Cloud, Microsoft Azure, Digital Ocean, etc. Dentre essas, há diferenças
de produtos oferecidos, preços, dentre outros fatores. Não será discutido aqui
especificidades de cada um.

Para o problema da reserva de ingressos, poderia ser usado, por exemplo, a AWS
que proporciona autoscaling \cite{aws-autoscaling} juntamente com o
Kubernetes \autoref{kubernetes}.
Desta forma, toda a infraestrutura seria escalável numa plataforma em nuvem
que hoje é considerada uma das maiores do mundo.

% \subsubsection{@AWS}
%
% Amazon Web Services (AWS)
%
% \url{https://aws.amazon.com/pt/}
%
% \url{https://aws.amazon.com/pt/blogs/compute/kubernetes-clusters-aws-kops/}
%
% \url{https://aws.amazon.com/pt/what-is-aws/}
%
% \url{https://aws.amazon.com/pt/about-aws/}
%
% tente focar no que ela pretende resolver (infraestrutura) e principalmente
% fazer links com o nosso tcc
%
% \subsubsubsection{@Auto Scaling na AWS}
% pode falar de produtos como Auto Scaling
% \url{https://aws.amazon.com/pt/autoscaling}
%
% e procura e me pergunta sobre outros assuntos relevantes pro tcc
%
%
%
% \subsubsection{@Google Cloud Platform (GCP)}
%
% nesse link também tem outros assuntos relacionados do lado direito
% \url{https://cloud.google.com/docs/overview/?hl=pt-br}
%
% você pode focar mais na AWS e não tanto no google plataform (se julgar necessario)
%
%
% \url{http://selbielabs.com/cloud-platforms-compared/}
%
% \subsubsection{@Heroku}
%
% \url{https://devcenter.heroku.com/}
%
% \subsubsection{@Digital Ocean}
%
% \subsubsection{@Microsoft Azure}
%
