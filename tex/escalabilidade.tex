\chapter{Escalabilidade}

\begin{citacao}

À medida que o volume de dados cresce, aumenta a
necessidade de escalabilidade e melhoria de desempenho. Dentre as soluções para este
problema, pode-se citar a escalabilidade vertical, que consiste em aumentar o poder de
processamento e armazenamento das máquinas, e a escalabilidade horizontal, onde ocorre um
aumento no número de máquinas disponíveis para o armazenamento e processamento de
dados. Em comparação com a escalabilidade vertical, a escalabilidade horizontal tende a ser
uma solução mais viável, porém requer que diversas threads/processos de uma tarefa sejam
criadas e distribuídas. \cite[3]{alexandre-morais-souza-2013}.

\end{citacao}

A escalabilidade horizontal é preferível por diversos fatores - principalmente por
permitir que seja possível ampliar a capacidade de processamento de requisições
apenas aumentando a quantidade de nós (unidades de processamento/computadores).
Além do que, dada uma arquitetura preparada, é possível fazer a automação da
arquitetura para que permita que novos nós sejam adicionadas enquanto o
sistema está recebendo requisições, tornando possível que aumente a capacidade
de processamento \cite{ivens-oliveira-porto-2009}.

Quando há um aumento significativo de processamento, memória, threads, etc,
é possível que novas máquinas sejam incluídas como nós de processamentos.
Isso é possível devido ao monitoramento das máquinas existentes na arquitetura
da aplicação.

Por exemplo, um parâmetro de máximo de processamento de 80\% é configurado.
Quando a aplicação atingir esse parâmetro, outro nó poderá ser iniciado para ajudar
a atender as demandas do sistema. Da mesma forma, pode ser configurada uma utilização
mínima. Por exemplo, quando o processamento chegar à 20\% de utilização, uma máquina
deve ser finalizada, economizando custos de infraestrutura.




%%%%%%%%%%%%%%%%%%%%%%%%%%%%

% daqui pra baixo são apenas comentários que fiz:

% \url{https://github.com/lhzsantana/escalabilidade}
%
% \section{Apache Kakfa}
% Apache Kakfa, para queueing
% \url{https://medium.com/@gabrielqueiroz/o-que-%C3%A9-esse-tal-de-apache-kafka-a8f447cac028}
%
% \section{Apache Spark}
% Apache Spark, para streaming
% \url{https://www.infoq.com/br/presentations/spark-streaming-kafka-sem-perdas}
%
% \section{Apache Cassandra}
% Apache Cassandra, para registro de transações

% vou ver exatamente como escrever sobre os seguintes assuntos:
%
% java usando técnicas de imutabilidade
% Rest para requisições síncronas
% Kafka para gerenciar fila de processamento (garantir ordem)
% Websocket para receber notificações (mosquitto

