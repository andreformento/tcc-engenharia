\subsection{Orquestração de contêiners com Kubernetes}\label{kubernetes}

O recurso utlizado para orquestração dos conteiners é o Kubernetes, ao longo desse 
tópico será abordada sua definição a função que ele exerce dentro da infraestrutura
bem como sua importância dentro do cenário.

\begin{citacao}
Orquestração de contêiners é a capacidade de provisionar automaticamente
a infraestrutura necessária para atender às solicitações das aplicações web
por meio de contêiners do Docker \cite{docker-kubernetes-e-openshift}.
\end{citacao}

Através da ferramenta Kubernetes é possível criar cluster (\autoref{cluster})
de aplicações em nuvem de maneira que seja fácil o seu gerenciamento.
Abaixo são listadas as vantagens em usar o Kubernetes:

\begin{itemize}

	\item Se houver algum problema no contêiner, o Kubernetes identifica e recria -
          mantendo assim a disponibilidade do serviço;
    \item Garante estabilidade das aplicações;
    \item Através de métricas provisiona novas instâncias da aplicação,
          proporcionando dessa forma auto escalabilidade de maneira horizontal;
    \item Aproveitamento melhor da máquina em que roda a aplicação - que em alguns
          casos pode rodar mais de um tipo de serviço, dependendo da estratégia
          que a infraestrutura utilizar;
    \item Possui um balanceador de carga para dividir as requisições entre
          aplicações instanciadas;
    \item Quando há a implantação de uma nova versão do serviço e for identificado
          algum problema, é possível fazer o rollback de maneira simples.

\end{itemize}

Com o uso de contêiners Dockers e o seu gerenciamento através do Kubernetes é possível
obter uma infraestrutura robusta, permitindo assim a escalabilidade necessária que permite
grande poder de processamento e atendimento de muitas requisições.
