\subsection{Orquestração de containers com Kubernetes}\label{kubernetes}

\begin{citacao}
Orquestração de containers é a capacidade de provisionar automaticamente
a infraestrutura necessária para atender às solicitações das aplicações web
por meio de containers do Docker \cite{docker-kubernetes-e-openshift}.
\end{citacao}

Através da ferramenta Kubernetes é possível criar cluster (\autoref{cluster})
de aplicações em nuvem de maneira que seja fácil o seu gerenciamento.
Abaixo são listadas as vantagens em usar o Kubernetes:

\begin{itemize}

	\item Se houver algum problema no container, o Kubernetes identifica e recria -
          mantendo assim a disponibilidade do serviço;
    \item Garante estabilidade das aplicações;
    \item Através de métricas provisiona novas instâncias da aplicação,
          proporcionando dessa forma auto escalabilidade de maneira horizontal;
    \item Aproveitamento melhor da máquina em que roda a aplicação - que em alguns
          casos pode rodar mais de um tipo de serviço, dependendo da estratégia
          que a infraestrutura utilizar;
    \item Possui um balanceador de carga para dividir as requisições entre
          aplicações instanciadas;
    \item Quando há a implantanção de uma nova versão do serviço e for identificado
          algum problema, é possível fazer o rollback de maneira simples.

\end{itemize}

Com o uso de containers Dockers e o seu gerenciamento através do Kubernetes é possível
obter uma infraestrutura robusta, permitindo assim a escalabilidade necessária que permite
grande poder de processamento e atendimento de muitas requisições.

%@manual{docker-kubernetes-e-openshift,
%	Author = {Gustavo Costa},
%	Date-Added = {2017-12-03 17:37:50 +0000},
%	Date-Modified = {2017-12-03 17:37:50 +0000},
%	Organization = {Concrete Solutions},
%	Title = {As diferenças entre Docker, Kubernetes e Openshift},
%	Url = {https://www.concrete.com.br/2017/06/26/docker-kubernetes-e-openshift/},
%	Urlaccessdate = {03 dez. 2017},
%	Year = {2017}}
%
%@manual{microservices-with-kubernetes-and-docker,
%	Author = {Piotr Mińkowski},
%	Date-Added = {2017-12-03 17:37:50 +0000},
%	Date-Modified = {2017-12-03 17:37:50 +0000},
%	Organization = {DZone},
%	Title = {Microservices With Kubernetes and Docker},
%	Url = {https://dzone.com/articles/microservices-with-kubernetes-and-docker},
%	Urlaccessdate = {03 dez. 2017},
%	Year = {2017}}
