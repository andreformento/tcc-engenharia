\part{Resultados}

%Foi possível demonstrar uma aplicação que escala sob demanda
%Foi apresentada uma arquitetura que é possível de ser aplicada em produção

\section{@Como inicializar a aplicação}

\subsubsection{@Requisitos}

\section{@Métricas}

\subsection{@Como gerar métricas}

\subsubsection{@Requisitos}

\section{Informações de Hardware do computador que foi feito o teste}

\index{alíneas}\index{subalíneas}\index{incisos}Os testes foram realizados em um notebook com as configurações conforme
descrito nos próximos tópicos.

\begin{alineas}

  \item CPU

  \begin{alineas}

     \item Arquitetura: x86\_64

     \item Empresa: Intel Corp.

     \item Produto: Intel(R) Core(TM) i7-4.800MQ CPU @ 2,70GHz

     \item Quantidade de núcleos: 8

     \item Frequência padrão: 2.848MHz

     \item Frequência mínima:  800MHz

     \item Frequência máxima: 3.700MHz

     \item Virtualização: VT-x

     \item L1d cache: 32K

     \item L1i cache: 32K

     \item L2 cache: 256K

     \item L3 cache: 6.144K

  \end{alineas}

  \item Memória

  \begin{alineas}

     \item Capacidade por unidade: 8.192 MB

     \item Quantidade: 2

     \item Capacidade total: 16.384 MB

     \item Modelo: SODIMM

     \item Set: None

     \item Tipo: DDR3

     \item Velocidade: 1.600 MT/s

  \end{alineas}

  \item Armazenamento

  \begin{alineas}

     \item Modelo: Samsung SSD 840

     \item Empresa: Samsung

     \item Device: SSD 840

     \item Detalhe: Samsung SSD 840 PRO Series (DXM05B0Q)

     \item Capacdade: 256 GB

  \end{alineas}

  \item Sistema operacional

  \begin{alineas}

     \item Sistema operacional: Linux

     \item Distro: Fedora

     \item Release: 26

     \item Versão do Kernel: 4.13.12-200.fc26.x86\_64

  \end{alineas}

\end{alineas}
