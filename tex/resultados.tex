\chapter{Infraestrutura}
%Foi possível demonstrar uma aplicação que escala sob demanda
%Foi apresentada uma arquitetura que é possível de ser aplicada em produção

\section{Uso da infraestrutura}

A escolha da infraestrutura para a solução final foi o uso da
computação em nuvem (\autoref{infraestrutura-em-nuvem}), visto que, proporciona
escalabilidade que o problema exige - uma das principais vantagens da computação em nuvem.
Além é claro, do baixo custo inicial que a aplicação terá antes de começar a gerar qualquer
tipo de lucro.

Conforme discutido a respeito das topologias das plataformas em
nuvem (\autoref{tipologia-das-plataformas-em-nuvem}), para a solução final foi escolhido o
modelo IaaS (\autoref{iaas}) que fornece a opção de rodar programas como Docker e
Kubernetes, tornando a escolha do serviço de plataforma em nuvem independente da forma
com que o software roda. Ou seja, há um desacoplamento da infraestrutura com o software
(\autoref{ferramentas}).

\section{Métricas}

Foram feitos vários testes em cima da aplicação para saber quais eram seus limites.
É importante deixar claro que as métricas ocorreram em um notebook pessoal com poder
de processamento limitado. Em um ambiente real (nuvem) poderão existir diversas máquinas
disponíveis para que sejam utilizadas (escalabilidade horizontal) permitindo que os
números exibidos abaixo sejam muito maiores.

\subsection{Ferramenta para geração de métricas - Gatling}

Os testes de perfomance realizados em cima da aplição de reserva de ingressos foram
feitos com a ferramenta Gatling na versão 2.3.
Com a ferramenta é possível testar alta performance de aplicações web
(\cite{gatling-docs}) que seguem o protocol HTTP.

\subsection{Especificações de teste}

\index{alíneas}\index{subalíneas}\index{incisos}As informações de Hardware
do computador que foram feitos os testes estão descritas nos próximos tópicos.
Os testes foram realizados em um notebook de uso pessoal.

\begin{alineas}

  \item CPU

  \begin{alineas}

     \item Arquitetura: x86\_64

     \item Empresa: Intel Corp.

     \item Produto: Intel(R) Core(TM) i7-4.800MQ CPU @ 2,70GHz

     \item Quantidade de núcleos: 8

     \item Frequência padrão: 2.848MHz

     \item Frequência mínima:  800MHz

     \item Frequência máxima: 3.700MHz

     \item Virtualização: VT-x

     \item L1d cache: 32K

     \item L1i cache: 32K

     \item L2 cache: 256K

     \item L3 cache: 6.144K

  \end{alineas}

  \item Memória

  \begin{alineas}

     \item Capacidade por unidade: 8.192 MB

     \item Quantidade: 2

     \item Capacidade total: 16.384 MB

     \item Modelo: SODIMM

     \item Set: None

     \item Tipo: DDR3

     \item Velocidade: 1.600 MT/s

  \end{alineas}

  \item Armazenamento

  \begin{alineas}

     \item Modelo: Samsung SSD 840

     \item Empresa: Samsung

     \item Device: SSD 840

     \item Detalhe: Samsung SSD 840 PRO Series (DXM05B0Q)

     \item Capacdade: 256 GB

  \end{alineas}

  \item Sistema operacional

  \begin{alineas}

     \item Sistema operacional: Linux

     \item Distro: Fedora

     \item Release: 26

     \item Versão do Kernel: 4.13.12-200.fc26.x86\_64

  \end{alineas}

\end{alineas}

\subsection{@@Resultados das métricas}



\section{@Trabalhos futuros}\label{trabalhos-futuros}

@@ Implementação de outros serviços; Implementação do frontend
