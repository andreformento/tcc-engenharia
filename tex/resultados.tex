\chapter{Infraestrutura}
%Foi possível demonstrar uma aplicação que escala sob demanda
%Foi apresentada uma arquitetura que é possível de ser aplicada em produção

\section{Uso da infraestrutura}

A escolha da infraestrutura para a solução final foi o uso da
computação em nuvem (\autoref{infraestrutura-em-nuvem}), visto que, proporciona
escalabilidade que o problema exige - uma das principais vantagens da computação em nuvem.
Além é claro, do baixo custo inicial que a aplicação terá antes de começar a gerar qualquer
tipo de lucro.

Conforme discutido a respeito das topologias das plataformas em
nuvem (\autoref{tipologia-das-plataformas-em-nuvem}), para a solução final foi escolhido o
modelo IaaS (\autoref{iaas}) que fornece a opção de rodar programas como Docker e
Kubernetes, tornando a escolha do serviço de plataforma em nuvem independente da forma
com que o software roda. Ou seja, há um desacoplamento da infraestrutura com o software
(\autoref{ferramentas}).

\section{Métricas}

Foram feitos vários testes em cima da aplicação para saber quais eram seus limites.
É importante deixar claro que as métricas ocorreram em um notebook pessoal com poder
de processamento limitado. Em um ambiente real (nuvem) poderão existir diversas máquinas
disponíveis para que sejam utilizadas (escalabilidade horizontal) permitindo que os
números exibidos abaixo sejam muito maiores.

\subsection{Gatling: Ferramenta para geração de métricas}

Os testes de perfomance realizados em cima da aplição de reserva de ingressos foram
feitos com a ferramenta Gatling na versão 2.3.
Com a ferramenta é possível testar alta performance de aplicações web
\cite{gatling-docs} que seguem o protocol HTTP.

A estratégia de teste de performance de requisições foi feita usando a
estratégia do método heavisideUsers \cite{gatling-simulation-setup}
que o Gatling possui, simulando assim a utilização do sistema por usuários.

\subsection{@@Resultados das métricas}



\section{@Trabalhos futuros}\label{trabalhos-futuros}

@@ Implementação de outros serviços; Implementação do frontend
