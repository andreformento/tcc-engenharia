%\chapter*[Introdução]{Introdução}
%\addcontentsline{toc}{chapter}{Introdução}

Sites que vendem ingresso online normalmente costumam agendar o início da venda
para um determinado horário.
Neste momento, vários usuários tentam realizar a compra ao mesmo tempo
- congestionando, ou até mesmo, derrubando o site.
A experiência costuma ser pior para eventos grandes, onde há muita procura.

O trabalho usa uma abordagem de microserviços e implementa uma parte deste
processo - que é o serviço de reserva do ingresso.
Este serviço roda em um servidor que pode estar na nuvem.
É nele que fica a lógica de controle do limite de ingressos que podem
ser vendidos num determinado evento.
Este controle deve ser feito de tal maneira que não prejudique a utilização
do sistema.
Ou seja, mesmo em momentos de grande demanda, o sistema deve continuar
respondendo de maneira satisfatória - mantendo a performance e a disponibilidade.
Por isso, será explicada a importância da escalabilidade horizontal,
junto com balanceamento de cargas.

Serão gerados resultados para demonstrar aspectos de desempenho da aplicação
implementada e limites técnicos que uma máquina servidora pode ter. Por fim,
serão mencionadas ferramentas, como Docker e Kubernetes, que ajudam na
implantação deste tipo de arquitetura.

% escalabilidade
% performance
% disponibilidade
% microserviço
% resultado
% balanceamento de cargas
% nuvem
