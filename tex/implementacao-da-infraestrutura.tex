\chapter{@Implementação da infraestrutura}\label{implementação-da-infraestrutura}

\section{Ferramentas}\label{ferramentas}

%Motivações (custo) de deixar desacoplarar infraestrutura de software

Há vários serviços que facilitam o desenvolvimento em de aplicações, como banco de dados,
gateways de api, mensageria, etc. Porém, a utilização destes serviços podem
tornar o produto dependente da plataforma onde está rodando. Por isso, é recomendável
a utilização de meios que deixem desacoplados os produtos da plataforma com o produto
digital desenvolvido.

\subsection{@Docker}\label{docker}

\url{http://www.diego-garcia.info/2015/02/15/docker-por-onde-comecar/}

\url{https://www.mundodocker.com.br}

\subsection{@Kubernetes}\label{kubernetes}

\url{https://www.concrete.com.br/2017/06/26/docker-kubernetes-e-openshift/}

\url{https://dzone.com/articles/microservices-with-kubernetes-and-docker}



\section{@Requisitos da aplicação}\label{requisitos-da-aplicacao}

% git, java, gradle, docker, kubernetes

\section{@Como rodar a aplicação}\label{como-rodar-a-aplicacao}





% lembrar que o conceito já foi explicado
% só deve ser informado como foram utilizadas as
% ferramentas
% uso do docker para subir a aplicação
% kubernetes para gerenciar o escalamento


