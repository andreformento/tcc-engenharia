\chapter{Implementação da infraestrutura}\label{implementação-da-infraestrutura}

\section{Ferramentas}\label{ferramentas}

%Motivações (custo) de deixar desacoplarar infraestrutura de software

Há vários serviços que facilitam o desenvolvimento em de aplicações, como banco de dados,
gateways de api, mensageria, etc. Porém, a utilização destes serviços podem
tornar o produto dependente da plataforma onde está rodando. Por isso, é recomendável
a utilização de meios que deixem desacoplados os produtos da plataforma com o produto
digital desenvolvido.

\subsection{@Docker}\label{docker}
Para a construção do trabalho, faz-se necessário abordar sobre uma ferramenta que auxilia na execução de aplicações web.
A ferramenta que se trata é o docker, segundo a definição do seu fundador Solomon Hykes. 
“O Docker é um conjunto de ferramentas para criar aplicativos distribuídos de forma muito específica. O foco da filosofia 
do projeto é a ideia de que os desenvolvedores podem escolher as ferramentas de que precisam para criar aplicativos da 
maneira que melhor se adequar às suas necessidades e preferências. Como resultado, nem todos os projetos usam o Docker
da mesma maneira.” \cite {solomon-hykes}.


Existe uma definição complementar. “Docker é uma plataforma Open Source escrita em Go, que é uma linguagem de
programação de alto desempenho desenvolvida dentro do Google, que facilita a criação e administração de ambientes isolados.”
\cite {mundodocker-01}.


Docker opera em ambiente virtual que possui como base os Contêineres Linux (LXC). “Contêineres Linux (LXC) é um tipo de
virtualização em nível de sistema operacional que proporciona a execução de múltiplas instâncias isoladas de um determinado
sistema operacional dentro de um único hospedeiro. O conceito é simples e antigo, sendo o comando chroot seu precursor mais
famoso. Com o chroot é possível segregar os acessos de diretórios e evitar que o usuário possa ter acesso à estrutura raiz 
(“/” ou root).” \cite {sinestec-01}.


A adoção do Docker está relacionada em atender uma demanda de desempenho de hardware e software. A justificativa para utiliza
r essa ferramenta quando se fala de hardware está no fato de evitar travamento dos equipamentos, agregar mais processamento 
quando necessário e desativá-los em baixo números de requisições.
O trecho seguinte descreve um dos problemas encontrados quando se utiliza máquinas virtuais em vez de contêineres. “Lembrando
que cada máquina virtual é um sistema operacional praticamente inteiro instalado sobre um equipamento físico e rodando para
servir a aplicação. Com certeza o servidor de desenvolvimento não teria recursos suficientes para rodar todos esses sistemas
simultaneamente, e nossa máquina física travaria".\cite {aprendendo-docker}.


Quando se questiona de software, as vantagens da utilização do docker estão em servir como instrumento versátil, que entrega
um ambiente pronto para rodar aplicações. Essa versatilidade pode ser aproveitada para iniciar serviços rapidamente e responder
a aplicação de forma assertiva.


“O Docker possibilita o empacotamento de uma aplicação ou ambiente inteiro dentro de um contêiner, e a partir desse momento o
ambiente inteiro torna-se portável para qualquer outro Host que contenha o Docker instalado.
Isso reduz drasticamente o tempo de deploy de alguma infraestrutura ou até mesmo aplicação, pois não há necessidade de ajustes
de ambiente para o correto funcionamento do serviço, o ambiente é sempre o mesmo, configure-o uma vez e replique-o quantas vezes
quiser”. \cite {mundodocker-02}


Como mencionado, a redução do tempo é um dos fatores percebidos ao adotar contêineres para execução de aplicações. A
escalabilidade também é satisfeita como descrito na citação abaixo.
“Um dos benefícios da abstração entre o sistema host e o contêiner é que, dado um projeto correto de aplicação, a 
escalabilidade pode ser simples e direta. Projeto orientado a serviço (discutido mais tarde) combinado com aplicações
conteinerizadas fornecem as bases para a escalabilidade fácil.
Um desenvolvedor pode executar alguns contêineres em sua estação de trabalho, enquanto este sistema pode ser escalado
horizontalmente em uma área de preparação ou teste. Quando os contêineres entram em produção, eles podem escalar novamente”. \cite {digitalocean-01}.

\subsection{@Kubernetes}\label{kubernetes}

Escrever sobre Kubernetes
\url{https://www.concrete.com.br/2017/06/26/docker-kubernetes-e-openshift/}

\url{https://dzone.com/articles/microservices-with-kubernetes-and-docker}



\section{Requisitos da aplicação}\label{requisitos-da-aplicacao}

Todas as operações mencionadas consideram que o Sistema Operacional é Linux
- que apesar de não ser obrigatório, é recomendado.
Para gerenciamento do código fonte é utilizado o Git no servidor do Github.
Para compilar e rodar a aplicação é necessário ter instalado o Java 8.
Para gerenciamento de pacotes do Java é utilizado o Gradle 4, mas não
é obrigatório que ele esteja instalado - pois é feito o download, caso
ele não exista. Para subir a aplicação é necessário ter o Docker e o
Kubernetes (kubectl e minikube) instalados.

% git, java, gradle, docker, kubernetes

\section{Como rodar a aplicação}\label{como-rodar-a-aplicacao}

Para fazer o download do código fonte é necessário utilizar a ferramenta Git
e executar no terminal do Linux o seguinte mostrado no \autoref{comando-git}.

\lstset{language=bash,keywordstyle={\bfseries \color{black}}}

\begin{lstlisting}[language=bash,label=comando-git,caption=Como fazer o download do código fonte com o Git]
git clone https://github.com/andreformento/term-paper.git
\end{lstlisting}

Para construir a aplicação é necessário executar o comando
\autoref{construir-aplicacao}.

\begin{lstlisting}[language=bash,label=construir-aplicacao,caption=Como construir a aplicação]
cd term-paper
./build.sh
\end{lstlisting}

Para rodar a aplicação é necessário executar o comando
\autoref{rodar-aplicacao}.

\begin{lstlisting}[language=bash,label=rodar-aplicacao,caption=Como rodar a aplicação]
./start.sh
\end{lstlisting}

Para testar a performance da aplicação é necessário executar o comando
\autoref{testar-performance-aplicacao}.

\begin{lstlisting}[language=bash,label=testar-performance-aplicacao,caption=Como testar a performance da aplicação]
./test-report.sh
\end{lstlisting}



% lembrar que o conceito já foi explicado
% só deve ser informado como foram utilizadas as
% ferramentas
% uso do docker para subir a aplicação
% kubernetes para gerenciar o escalamento


