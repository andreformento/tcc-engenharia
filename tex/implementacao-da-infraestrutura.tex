\chapter{Implementação da infraestrutura}\label{implementação-da-infraestrutura}

\section{Ferramentas}\label{ferramentas}

%Motivações (custo) de deixar desacoplarar infraestrutura de software

Há vários serviços que facilitam o desenvolvimento de aplicações, como banco de dados,
gateways de api, mensageria, etc. Porém, a utilização destes serviços pode
tornar o produto dependente da plataforma onde está rodando. Por isso, é recomendável
a utilização de meios que deixem desacoplados os produtos da plataforma com o produto
digital desenvolvido.

\subsection{Docker}\label{docker}

Foi adotada a ferramenta Docker para que fosse simples a execução do serviço de reserva de ingressos.

\begin{citacao}
O Docker é um conjunto de ferramentas para criar aplicativos distribuídos de forma muito específica.
O foco da filosofia do projeto é a ideia de que os desenvolvedores podem escolher as ferramentas
de que precisam para criar aplicativos da maneira que melhor se adequar às suas necessidades e preferências.
Como resultado, nem todos os projetos usam o Docker da mesma maneira \cite{solomon-hykes}.
\end{citacao}

Conforme descrito por \citeonline{mundodocker-o-que-e-docker}, ``[Docker] facilita a criação e administração de ambientes isolados``.
Ou seja, é uma forma de rodar um aplicação em um ambiente Linux com todos os recursos necessários sem interferir
diretamente na máquina que executa a aplicação - por causa do isolamento.

\begin{citacao}
Contêineres Linux (LXC) é um tipo de
virtualização em nível de sistema operacional que proporciona a execução de múltiplas instâncias isoladas de um determinado
sistema operacional dentro de um único hospedeiro \cite{sinestec-01}.
\end{citacao}

Conforme a citação anterior, é possível executar vários serviços dentro de uma mesma máquina, podendo ser serviços
replicados ou vários tipos diferentes de serviços.
Na aplicação de reserva de ingressos, que foi desenvolvida em Java e precisa de uma JVM para rodar, não precisaria
de nenhuma instalação na máquina de qualquer tipo de programa.
Tudo estaria instalado dentro da imagem Docker criada para rodar a aplicação.
Então, o banco de dados, ou outro serviço que a aplicação dependesse, também poderia rodar em um contêiner Docker.
Ficaria fácil até mesmo trocar toda a implementação para rodar em outra linguagem.

A justificativa para utilizar essa ferramenta também está relacionada a travamentos, rápida execução e desativaçao.
O contêiner é uma evolução em relação ao uso de máquinas virtuais.

\begin{citacao}
Lembrando que cada máquina virtual é um sistema operacional praticamente inteiro instalado sobre um equipamento físico e
rodando para servir a aplicação. Com certeza o servidor de desenvolvimento não teria recursos suficientes para rodar
todos esses sistemas simultaneamente, e nossa máquina física travaria\cite{aprendendo-docker}.
\end{citacao}


Quando se questiona de software, as vantagens da utilização do docker estão em servir como instrumento versátil, que entrega
um ambiente pronto para rodar aplicações. Essa versatilidade pode ser aproveitada para iniciar serviços rapidamente e responder
a aplicação de forma assertiva.


\begin{citacao}
O Docker possibilita o empacotamento de uma aplicação ou ambiente inteiro dentro de um contêiner, e a partir desse momento o
ambiente inteiro torna-se portável para qualquer outro Host que contenha o Docker instalado.
Isso reduz drasticamente o tempo de deploy de alguma infraestrutura ou até mesmo aplicação, pois não há necessidade de ajustes
de ambiente para o correto funcionamento do serviço, o ambiente é sempre o mesmo, configure-o uma vez e replique-o quantas vezes
quiser \cite{mundodocker-o-que-e-docker}.
\end{citacao}

Como mencionado, a redução do tempo é um dos fatores percebidos ao adotar contêineres para execução de aplicações.
A escalabilidade também é satisfeita como descrito na citação abaixo.

\begin{citacao}
Um dos benefícios da abstração entre o sistema host e o contêiner é que, dado um projeto correto de aplicação, a
escalabilidade pode ser simples e direta. Projeto orientado a serviço combinado com aplicações
conteinerizadas fornecem as bases para a escalabilidade fácil.
Um desenvolvedor pode executar alguns contêineres em sua estação de trabalho, enquanto este sistema pode ser escalado
horizontalmente em uma área de preparação ou teste.
Quando os contêineres entram em produção, eles podem escalar novamente
\cite{o-ecossistema-do-docker-uma-visao-geral-da-conteinerizacao-pt}.
\end{citacao}


\subsection{@Kubernetes}\label{kubernetes}

Escrever sobre Kubernetes

\url{https://www.concrete.com.br/2017/06/26/docker-kubernetes-e-openshift/}

\url{https://dzone.com/articles/microservices-with-kubernetes-and-docker}


\section{Requisitos da aplicação}\label{requisitos-da-aplicacao}

Todas as operações mencionadas consideram que o Sistema Operacional é Linux
- que apesar de não ser obrigatório, é recomendado.
Para gerenciamento do código fonte é utilizado o Git no servidor do Github.
Para compilar e rodar a aplicação é necessário ter instalado o Java 8.
Para gerenciamento de pacotes do Java é utilizado o Gradle 4, mas não
é obrigatório que ele esteja instalado - pois é feito o download, caso
ele não exista. Para subir a aplicação é necessário ter o Docker e o
Kubernetes (kubectl e minikube) instalados.

% git, java, gradle, docker, kubernetes

\section{Como rodar a aplicação}\label{como-rodar-a-aplicacao}

Para fazer o download do código fonte é necessário utilizar a ferramenta Git
e executar no terminal do Linux o seguinte mostrado no (\autoref{comando-git}).

\lstset{language=bash,keywordstyle={\bfseries \color{black}}}

\begin{lstlisting}[language=bash,label=comando-git,caption=Como fazer o download do código fonte com o Git]
git clone https://github.com/andreformento/term-paper.git
\end{lstlisting}

Para construir a aplicação é necessário executar o comando
(\autoref{construir-aplicacao}).

\begin{lstlisting}[language=bash,label=construir-aplicacao,caption=Como construir a aplicação]
cd term-paper
./build.sh
\end{lstlisting}

Para rodar a aplicação é necessário executar o comando
(\autoref{rodar-aplicacao}).

\begin{lstlisting}[language=bash,label=rodar-aplicacao,caption=Como rodar a aplicação]
./start.sh
\end{lstlisting}

Para testar a performance da aplicação é necessário executar o comando
(\autoref{testar-performance-aplicacao}).

\begin{lstlisting}[language=bash,label=testar-performance-aplicacao,caption=Como testar a performance da aplicação]
./test-report.sh
\end{lstlisting}



% lembrar que o conceito já foi explicado
% só deve ser informado como foram utilizadas as
% ferramentas
% uso do docker para subir a aplicação
% kubernetes para gerenciar o escalamento


