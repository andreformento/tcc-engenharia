\chapter{Disponibilidade}

A disponibilidade tem como objetivo manter o maior tempo possível
o sistema respondendo requisições. No contexto de reserva de ingressos,
significa que a qualquer momento um cliente pode solicitar a reserva para
um determinado evento e ele deve ser atendido. Da mesma forma, quando houver
um aumento de demanda, o sistema não deveria sofrer qualquer dano. A alta
disponibilidade provê uma forma de minimizar possíveis falhas em determinados
serviços.

Antes de entender alta disponibilidade é necessário entender os clusters.
Eles servem como uma técnica de tolerência a falhas e tem como função a
distribuição, redundância e homogeneidade.

\section{Cluster}\label{cluster}

Um cluster pode ser definido como um grupo de computadores que juntos
criam um sistema mais poderoso que apenas uma máquina. O conjunto
independente de máquinas cooperam entre si para atender um objetivo comum.

No caso do problema de reserva de ingressos, é possível que hajam vários
computadores que trabalhem em conjunto para deixar no ar o serviço.
Ou seja, mesmo que ocorra algum problema com uma máquina, haverão outras
disponíveis atendendo as requisições.
Pode haver um monitoramento que valide as máquinas que deixam de funcionar
e colocar outra no lugar - tudo isso de maneira automatizada.
Desta forma, é possível criar um sistema robusto que mantenha um desempenho
constante durante a sua execução.
Desta forma, as interrupções muitas vezes não impactam os processos existentes.

\section{Alta disponibilidade}

Um conjunto de clusters pode ser utilizado para paralelizar as requisições
que um determinado serviço deve atender, gerando assim a alta disponibilidade.

\begin{citacao}

Alta disponibilidade - quando desejamos que o cluster forneça determinados
serviços que devem estar sempre (ou quase sempre) disponíveis para receber
solicitações.
Este nível de disponibilidade do serviço é um fator dependente do cluster
\cite{servicos-de-pertinencia-para-clusters-de-alta-disponibilidade}.

\end{citacao}

\section{Cálculo de tolerência a falhas}

Esses mecanismos tem que como consequência um sistema que opera
muitas vezes em mais de 99,99\% do tempo.
Um cálculo anual é feito para conhecer o tempo de inatividade do componente.
A fórmula possui como parâmetro o grau de tolerância e tempo médio entre falhas.
Os cálculos são feitos conforme a \autoref{equacao-disponibilidade}.

\begin{equation}\label{equacao-disponibilidade}
Disponibilidade = \frac{MTBF}{MTBF + MTTR}
\end{equation}

\index{alíneas}\index{subalíneas}\index{incisos}Onde:

\begin{itemize}

	\item MTBF: Mean Time Between Failures - tempo médio entre falhas

	\item MTTR: Mean Time To Repair - tempo médio de reparação

\end{itemize}

\section{Alta eficiência}

Garantir 99\% do tempo disponível significa até 3,65 dias de inatividade por ano.
Organizações que trabalham com serviços web priorizam o máximo do tempo on-line e buscam
serviços que entregam o menor tempo de inatividade possível.
Para atender essa necessidade, empresas investem em infraestrutura e desempenham um alto
grau de operação, atingindo 99,999\% do tempo de atividade por ano - um downtime/ano de
apenas 5 horas.

Empresas que possuem negócios on-line, são as que mais requisitam esse serviço.
Reduzir eventos como falhas e interrupções agregam mais confiança e fidelidade dos clientes.

Para que seja possível a alta escalabilidade é necessário que consiga absorver milhões de
requisições por segundo que podem ser oriundas de todas as partes do planeta.
Constantes indisponibilidades em aplicações causam perdas de clientes ou tarefas e a experiência
ruim do usuário fica atralada a marca.

% Disponibilidade pode ser entendida como um conjunto
%
% O sistema que se usa em alta disponibilidade pode ser o resultado de um conjunto de subsistemas que são capazes de executar
% funções individuais, porém sua principal utilidade é integrar e auxiliar os demais subsistemas que garantem uma operação única,
% um comportamento similar aos sistemas que se utiliza em ou pequenas ou médias empresas, mais são compostos de apenas um sistema
% que gerencia todos os componentes e demais softwares.
%
%
% O sistema operacional que atua em alta disponibilidade, é projetado para além de outras funções, monitorar o hardware, a fim
% de tomar decisões que redirecione as requisições, caso intercepte falhas que comprometam a execução da tarefa. Esse
% monitoramento é um dos pilares que garantem e classificam um sistema altamente disponível.
% O sistema oferece aos seus utilizadores serviços. Estes são disponibilizados e executam uma série de instruções requisitadas
% pelo operador.
%
%
% Apesar de trabalhar com o intuito principal de evitá-las, esses sistemas ainda estão passiveis de falhas. Repostas não
% esperadas, interrupções físicas, eventos inesperados que podem estar no hardware e afetam o funcionamento interferindo na
% disponibilidade. Cada ocorrência que não esta dentro do que é esperado é classificada como falha. Apesar de não esperada,
% ela deve ser corrigida com o menor tempo possível. Sua origem esta relacionada com falhas humanas ou não, esses eventos
% incorretos depois de corrigidos passam a ser tratados de forma a evitar-se no futuro. Depois de armazenar essas falhas,
% elabora-se relatórios que apontam o tempo médio de falhas existentes. Todo hardware está fadado ao envelhecimento, o tempo
% médio de falhas aponta o quanto esse sistema está passível de quedas, analistas utilizam esses resultados para manutenção
% preventiva ou na aquisição de novos componentes.
%
%
% Existe também outro tempo mensurado para conhecer-se as discrepâncias do sistema. Somando-se todas as diferenças entre tempo
% desligado e o tempo ligado, dividindo pelo número de falhas obtidos anteriormente, consegue-se obter o tempo médio entre falhas.
% Utiliza-se esse tempo para se ter uma previsão do tempo de atividade do sistema em sua vida útil. Essa previsão faz parte de uma
% análise que conseguirá avaliar casos que reparação ou até mesmo renovação do conjunto.
% As falhas que exitem podem ter sua origem a partir de um defeito. Utiliza-se o termo defeito quando o sistema apresenta
% resultados inesperados dentro de uma análise realizada pelo usuário. Trava-se a imagem, exibe-se dados incorretos ou mensagens
% erros, maneiras que o conjunto usa para alertar o usuário que algo esta errado. Certas situações não são percebidas de imediato,
% estas permanecem nos sistemas causando latências e inatividade.
%
%
% O estado errôneo ou em erro, pode ser percebido quando se tem no processamento posterior um defeito. Este estado está
% relacionado em um universo informacional. Decorrente de uma falha o sistema apresenta informações incoerentes ao resultado
% esperado. O comportamento não atende as exigências de um funcionamento adequado causando-se defeitos que serão observados
% pelo operador. Entende-se que a classificação dos estados citados são estados de falha, que abrange toda parte física, estado
% de erro o que compete ao universo informacional e o estado de defeito que está relacionado em um cenário controlado pelo
% usuário.
%
%
% De tal relevância no assunto, as falhas foram ao logo de tempo muito estudadas, concluiu-se que dentro do estado de falha
% é possível identificar tipos diferentes desse comportamento.
% Distingue-se as falhas por duração como transientes ou permanentes. A primeira possui tempo de duração curto, suas causas
% estão atreladas ao mal funcionamento ou alguma interferência externa indesejada. Elas também podem ser intermitentes,
% afetando a operação em curtos porém diversos momentos distintos.
% Falhas que causam problemas de forma a não operar todo o hardware com consistência, eventos afetam de tal maneira que o
% desempenho é comprometido permanentemente, conhecidas como falhas permanentes, estas impossibilitam o sistema a funcionar
% como antes do ocorrido.
% Passível de outras falhas que competem a algum componente específico, conhece-se essas falhas como;
% De travamento, bem intuitivo esta falha representa algum travamento causado por um componente que apresenta defeito e
% perde-se do seu estado interno normal.
%
%
% Quando um componente responde requisições que a ele foram solicitadas, de forma antecipada ou atrasada, ele está fora
% de sincronismo com os demais, chama-se esse evento de falhas de timing ou falhas de performance.
% Falhas por omissão se dão por conta de um processo que ao entrar em colapso não conseguiu executar o próximo passo do
% programa. A não execução foi repentina, ou seja, quando foi invocado o processo ele simplesmente não respondeu. Outros
% processos identificam a falha por omissão através de timeouts, as requisições possuem tempo de espera, caso esse período
% seja ultrapassado, considera-se esse estado com falha.
%
%
% Caso o sistema opere de forma assíncrona, esta falha pode ser tratada como uma reposta ainda não recebida ou lenta. Para
% que o colapso da falha possa ser identificado, outros processos precisam detectar essa ocorrência, considerando o timeout
% de cada processo envolvido. Diferente do anterior, o sistema síncrono tem por sua vez os períodos idênticos de reposta de
% cada processo, caso não responde nesse tempo preestabelecido, concluiu-se falha na execução.
%
%
% Outro ponto a ser abordado no que se refere a falhas por omissão, são as falhas por comunicação. Considere-se os princípios
% simples de comunicação, o envio e recebimento de mensagens também possuem falhas. O fluxo inicia-se enviando uma mensagem,
% está é armazenada no buffer do recebedor. Quando o processo identifica que o buffer possui novas entradas, recupera essa
% mensagem. Quando ele não se concretiza por completo, o canal de comunicação determina falha de omissão por comunicação.
%
%
% Falhas arbitrárias ou bizantinas – avalia-se como uma das piores situações de falhas dentro do sistema. Isso porque é de
% difícil detecção, já que seus sintomas não são visíveis tão facilmente. Nesse estado, pode-se assumir valores diferentes
% dos seus dados, retornando um resultado errado mesmo que as vezes próximo do esperado. Noutro exemplo, falhas arbitrarias
% podem surgir afetando os canais de comunicação com mensagens corrompida ou inconsistentes.
