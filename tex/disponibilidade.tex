\chapter{Disponibilidade}

Alta disponibilidade

Muito utilizado em tecnologia, alta disponibilidade é descrita como a maneira de identificar, corrigir e evitar falhas ao longo do tempo de operação do sistema. Visa atender o máximo de tempo possível as requisições solicitadas. Para que isso seja alcançado, é necessária uma infraestrutura complexa, uma redundância nos equipamentos físicos e uma orquestração de plataformas que devem estar em harmônia com o hardware. Sistema que gerenciam esse conjunto de equipamentos são desenvolvidos de maneira específica, eles são adequados para que consigam usar todos os recursos que o hardware possa oferecer. Normalmente o software é construído em paralelo com o hardware, porém atribuindo ao kernel configurações existentes que compõem os equipamentos. Caso haja alguma anomalia ou falhas que comprometam o sistema, este é capaz de abortar sessões onde estão ocorrendo os problemas e disponibilizar outro ambiente livre que atenda o operador, ou seja, a interrupção muitas vezes não impactam os processos existentes.
Toda essa complexidade resulta em um sistema que opera muitas vezes em mais de 95% do tempo. Esse período é medido anualmente, para conhecer o tempo de inatividade do componente utiliza-se uma fórmula, esta possui como parâmetros, graus de tolerância e tempo médio entre falhas. Os cálculos são feitos com a equação abaixo;
MTBF- Mean Time Between Failures - (tempo médio entre falhas)
MTTR- Mean Time To ReMpair - (tempo médio de recuperação)
Disponibilidade = MTBF / (MTBF + MTTR)
Um sistema que garante 99% do tempo anualmente tolera até 3,65 dias por ano de inatividade. Números atrativos para muitas empresas, porém organizações que atuam na internet exigem um tempo disponível mais alto do que o exemplificado. Organizações que trabalham com serviços web, priorizam o máximo do tempo on-line e buscam serviços que entregam o menor tempo de inatividade possível. Para atender essa necessidade, empresas investem em maquinários e infraestrutura que tem um custo elevado, mais desempenham um alto grau de operação, atingindo 99,999% do tempo de atividade por ano, um downtime/ano de apenas 5 horas.
Empresas que possuem negócios on-line, são as que mais requisitam esse serviço. Reduzir eventos como falhas e interrupções agregam mais confiança e fidelidade dos clientes, por esta razão, muitos negócios estão sendo migrados para essa plataforma. Fora do âmbito tecnológico, outras empresas estão se tornando cada vez mais globalizadas. Essas mudanças implicam novos desafios, um deles é a criticidade de acessar as informações em onde quer que estejam. Visto que isso é uma demanda crescente a alta disponibilidade está é uma alternativa que atende qualquer ramo de negócio, seja ele pequeno, médio ou de grande porte.
Com as diversas vantagens apresentadas, a alta disponibilidade tem seu preço. Os benefícios podem ser aproveitados, porém estão atrelados pela necessidade ainda maior de um hardware robusto que consiga absorver milhões de requisições por segundo. Essas requisições são oriundas de todas as partes do planeta e precisam ser atendidas no menor tempo disponível. Constantes indisponibilidades em aplicações que são críticas, causam perdas de clientes ou tarefas e a imagem negativa que é associada a empresa torna-se duradoura e difícil de apagar.
Algumas terminologias usadas em alta disponibilidade são as mesmas que se usa em computação, com algumas mudanças. O sistema que se usa em alta disponibilidade pode ser o resultado de um conjunto de subsistemas que são capazes de executar funções individuais, porém sua principal utilidade é integrar e auxiliar os demais subsistemas que garantem uma operação única, um comportamento similar aos sistemas que se utiliza em ou pequenas ou médias empresas, mais são compostos de apenas um sistema que gerencia todos os componentes e demais softwares.
O sistema operacional que atua em alta disponibilidade, é projetado para além de outras funções, monitorar o hardware, a fim de tomar decisões que redirecione as requisições, caso intercepte falhas que comprometam a execução da tarefa. Esse monitoramento é um dos pilares que garantem e classificam um sistema altamente disponível.
O sistema oferece aos seus utilizadores serviços. Estes são disponibilizados e executam uma série de instruções requisitadas pelo operador.
Apesar de trabalhar com o intuito principal de evitá-las, esses sistemas ainda estão passiveis de falhas. Repostas não esperadas, interrupções físicas, eventos inesperados que podem estar no hardware e afetam o funcionamento interferindo na disponibilidade. Cada ocorrência que não esta dentro do que é esperado é classificada como falha. Apesar de não esperada, ela deve ser corrigida com o menor tempo possível. Sua origem esta relacionada com falhas humanas ou não, esses eventos incorretos depois de corrigidos passam a ser tratados de forma a evitar-se no futuro. Depois de armazenar essas falhas, elabora-se relatórios que apontam o tempo médio de falhas existentes. Todo hardware está fadado ao envelhecimento, o tempo médio de falhas aponta o quanto esse sistema está passível de quedas, analistas utilizam esses resultados para manutenção preventiva ou na aquisição de novos componentes. 
